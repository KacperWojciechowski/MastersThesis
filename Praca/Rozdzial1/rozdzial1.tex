\chapter{Wstęp}
\section{Wprowadzenie} % -------------------------------------------------------------------------

We współczesnym stanie techniki coraz częściej można spotkać się z urządzeniami i programami o inteligentnych funkcjach, takich jak predykcja zjawisk na podstawie zestawu danych, rozpoznawanie obrazu, analiza mowy, czy przetwarzanie języka naturalnego. Znajdują one zastosowanie w różnych dziedzinach codziennego życia, m.in. w medycynie. W zależności od potrzeb, techniki uczenia maszynowego można wykorzystać do zastosowań medycznych, jak np. rozpoznawanie komórek rakowych na skanach rezonansem magnetycznym, podejmowanie decyzji na podstawie zbioru objawów obecnych u pacjenta, lub przewidywanie norm związków naturalnie występujących w organiźmie ludzkim w zależności od okoliczności i wyników pomiarów. 

Jedną z istotnych dziedzin medycyny jest biostatystyka, polegająca na wykorzystaniu analizy statystycznej do wnioskowania na podstawie zbiorów danych, takich jak rezultaty przeprowadzonych badań (np. morfologicznych, poziomu poszczególnych hormonów we krwii, itp.), informacji o nawykach żywieniowych oraz stylu życia pacjenta. Szczególnie istotną formą systemów operujących w tej dziedzinie są systemy eksperckie, wykorzystujące techniki płytkiego i głębokiego uczenia maszynowego w celu wspierania diagnozy stawianej przez wykwalifikowanych lekarzy. 

U podstaw wyżej wymienionych zagadnień leży implementacja rozwiązań opartych o teorię uczenia maszynowego, oraz wszelkie związane z tym problemy. W związku z tym na przestrzeni lat powstało wiele gotowych narzędzi, takich jak biblioteki i \textit{frameworki}, mające na celu wsparcie programistów w szybkim i prawidłowym wprowadzaniu rozwiązań sztucznej inteligencji na różne platformy docelowe oraz w różnych językach, począwszy od języka C++, przez Python, po środowiska takie jak Matlab. 

Istotnym krokiem w przygotowywaniu oprogramowania wykorzystującego sztuczną inteligencję jest prawidłowy wybór wspomnianych wcześniej narzędzi dokonywany na etapie projektowania, tak, aby oferowały one możliwości adekwatne do wymagań funkcjonalnych. Niniejsza praca dokonuje analizy porównawczej bibliotek uczenia maszynowego dla języka C++ w kontekście zastosowań w dziedzinie biostatystyki, celem umożliwienia czytelnikowi trafnego wyboru odpowiedniego narzędzia do realizacji projektu badawczego.

\section{Cel i zakres pracy} % ----------------------------------------------------------------------

Celem pracy jest przeprowadzenie analizy i przygotwanie zestawienia bibliotek do uczenia maszynowego dla języka C++, obrazując przykłady bazujące na zestawie danych biostatystycznych.

Zakres pracy obejmował:

\begin{itemize}
    \item [$\bullet$] Przegląd dostępnych bibliotek języka C++;
    \item [$\bullet$] Inżynierię i kształtowanie danych;
    \item [$\bullet$] Płytkie i głębokie uczenie nadzorowane;
    \item [$\bullet$] Kwestie wydajnościowe w dopasowywaniu i wdrażaniu modeli;
    \item [$\bullet$] Badania praktyczne w oparciu o zestaw danych medycznych i biologicznych.
\end{itemize}

\section{Struktura pracy} % ----------------------------------------------------------------------

<<<<<<< HEAD
Pierwszy rozdział przedstawia ogólnym zagadnieniem dotykanym przez pracę, począwszy dziedziny problemu i jej zastosowań, do istoty tematu pracy. Dodatkowo omawiany jest cel i zakres realizacji pracy, oraz jej strukturą.

Kolejny rozdział wprowadza czytelnika do tematu uczenia maszynowego, oraz napotykanych w nim problemów dotyczących złożoności obliczeniowej oraz zużycia zasobów. Stanowią one podstawę do zaproponowania języka C++ jako technologi wspierającej ich rozwiązanie przy pomocy bibliotek. 

Tematem rozdziału trzeciego jest przygotowanie elementów testowych do wykorzystania w późniejszej analizie porównawczej. Składa się na nie wybranie i przygotowanie do zestawu danych biostatystycznych do procesu uczenia oraz wybrane wzorcowych rozwiązań. Czytelnik przeprowadzony jest przez normalizację danych i selekcję najlepiej dopasowanych regresorów, oraz zostaje zapoznany z przykładowymi wynikami rozwiązań wzorcowych.

Dalsza część pracy składa się z bloku omówienia i analizy wybranych bibliotek uczenia maszynowego pod kątem określonych kryteriów. Pierwszą z nich, opisaną w rozdziale czwartym, jest biblioteka TensorFlow autorstwa Google. W poszczególnych sekcjach poruszone zostały tematy takie jak możliwe sposoby wdrożenia utworzonych w niej modeli, wymagana forma danych, dostępne funkcjonalności związane z danymi oraz modelami, a także dostępność źródeł informacji na temat sposobu pracy z biblioteką.  
=======
Rozdział pierwszy wprowadza czytelnika do zagadnień związanych z uczeniem maszynowym oraz jego implementacją oraz przedstawia cel i zakres pracy, a także jej strukturę.

Następny rozdział stanowi omówienie praktycznych problemów implementacyjnych, takich jak ograniczenia pamięciowe oraz złożoność obliczeniowa. Następnie zaproponowane zostaje narzędzie pozwalajace na rozwiązanie powyższych ograniczeń w postaci języka C++, oraz omówione i uzasadnione zostaje istnienie bibliotek. 

Kluczowym temat rozdziału trzeciego stanowi przygotowanie stosownego zestawu danych i wybranych na ich podstawie rozwiązań szablonowych, do wykorzystania jako punkt odniesienia w testach omawianych w niniejszej pracy bibliotek. Poszczególne sekcje przeprowadzają czytelnika przez charakterystykę wybranych danych, sposób ich czyszczenia i normalizacji, oraz strukturę wybranych metod testowych. 

DO NAPISANIA DALEJ
>>>>>>> 0307b36fa8a25267e42af207b55a1d399d59cce1
