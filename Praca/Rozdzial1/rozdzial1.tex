\chapter{Wstęp}
\section{Wprowadzenie} % -------------------------------------------------------------------------

We współczesnym stanie techniki coraz częściej można spotkać się z urządzeniami i programami o inteligentnych funkcjach, takich jak predykcja na podstawie zestawu danych, rozpoznawanie obrazu, rozpoznawanie mowy i przetwarzanie języka naturalnego. Znajdują one zastosowanie w różnych dziedzinach codziennego życia, m.in. w medycynie. W zależności od potrzeb techniki uczenia maszynowego można wykorzystać do zastosowań medycznych dla zadań takich jak np. rozpoznawanie komórek rakowych na skanach rezonansem magnetycznym, podejmowanie decyzji na podstawie zbioru objawów obecnych u pacjenta, lub przewidywanie norm związków naturalnie występujących w organiźmie ludzkim w zależności od okoliczności. 

Jedną z istotnych dziedzin medycyny jest biostatystyka, polegająca na wykorzystaniu analizy statystycznej do wnioskowania na podstawie zbiorów danych, takich jak wyniki morfologiczne, wyniki innych przeprowadzonych badań, informacji o nawykach żywieniowych i stylu życia pacjenta. Szczególnie istotną formą systemów operujących w tej dziedzinie są systemy eksperckie, wykorzystujące techniki płytkiego i głębokiego uczenia maszynowego w celu wspierania diagnozy stawianej przez wykwalifikowanych lekarzy. 

U podstaw wyżej wymienionych zagadnień leży implementacja rozwiązań opartych o teorię uczenia maszynowego, oraz wszelkie związane z tym problemy. W związku z tym na przestrzeni lat powstało wiele gotowych bibliotek i \textit{framework}'ów mających na celu wsparcie programistów w szybkim i prawidłowym wprowadzaniu rozwiązań sztucznej inteligencji w różnych językach, począwszy od języka C++, przez Python, po środowiska takie jak Matlab. 

Istotnym krokiem w przygotowywaniu produktu wykorzystującego sztuczną inteligencję jest wybór narzędzi dokonywany na etapie projektowania, tak, aby oferowały one możliwości adekwatne do wymagań funkcjonalnych przygotowywanego produktu. Niniejsza praca dokonuje analizy porównawczej bibliotek uczenia maszynowego dla języka C++ w kontekście zastosowań w dziedzinie biostatystyki.

\section{Cel i zakres pracy} % ----------------------------------------------------------------------

Celem pracy jest przeprowadzenie analizy i przygotwanie zestawienia bibliotek do uczenia maszynowego dla języka C++, obrazując przykłady bazujące na zestawie danych biostatystycznych.

Zakres pracy obejmował:

\begin{itemize}
    \item [$\bullet$] Przegląd dostępnych bibliotek języka C++;
    \item [$\bullet$] Inżynierię i kształtowanie danych;
    \item [$\bullet$] Płytkie i głębokie uczenie nadzorowane;
    \item [$\bullet$] Kwestie wydajnościowe w dopasowywaniu i wdrażaniu modeli;
    \item [$\bullet$] Badania praktyczne w oparciu o zestaw danych medycznych i biologicznych.
\end{itemize}

\section{Struktura pracy} % ----------------------------------------------------------------------
DO NAPISANIA