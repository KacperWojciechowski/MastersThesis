\chapter{Introduction}
\section{About modern challenges} % -------------------------------------------------------------------------

In current times people encounter more and more devices with intelligent functions, such as predicting phenomenons based on various datasets, image recognition, speech analysis, or natural language processing. They make its way into more daily life aspects, such as eg. medicine, work and private homes. Depending on the needs, few of the applicable uses of machine learning in medicine are recognition of cancer cells using MRI scans, diagnosis based on patient's symptoms, or setting norms for compounds naturally occuring in human body, depending on the situation (such as stress) and organic samples tests.

One of the major parts of medical field is biostatistics, based on statistical analysis and reasoning based on data such as blood tests results, eating habits and lifestyle of the patient. Especially important type of systems present in this field are expert systems, utilizing shallow and deep machine learning to help doctors diagnose any potential issues. \cite{expert}

At the foundation of aforementioned subjects lies the implementation of machine learning solutions and all related problems. Due to this fact, over the decades specialists created various tools, such as libraries and frameworks, aiming to help programmers in quick and accurate application of artificial intelligence products on different platforms and in different languages, starting from C++, through Python, reaching environments such as Matlab \cite{tf_api}.

A major step in preparing AI software is proper selection of needed tools, performed at the design stage, so that they offer functionalities adequate to the requirements. This dissertation performs comparative analysis of machine learning libraries for C++ for use in biostatistics, and aims to help the reader choose the right tools for a research project. It is worth to mention, that only a selected subset of functionalities was presented, suitable for the context of test datasets. As such, mentioned libraries can offer broader range of more precise methods, or new method added after creation of this paper. The realization of this dissertation, the implementation parts were inspired by the examples in this handbook \cite{handsOnMachineLearning}, however each listing is the creation of the author, fine-tuned to the used methods and datasets, and equiped with evaluation mechanisms.

\section{Aim and scope} % ----------------------------------------------------------------------

Aim of this dissertation is to perform a comparative analysis and breakdown of machine learning libraries for C++, presenting examples based on biostatistics datasets.

Scope of the thesis:

\begin{itemize}
    \item [$\bullet$] overview of available libraries;
    \item [$\bullet$] data engineering;
    \item [$\bullet$] shallow and deep supervised learning;
    \item [$\bullet$] efficiency aspects of selecting models;
    \item [$\bullet$] practical experiment based on medical and biological data.
\end{itemize}

\section{Structure} % ----------------------------------------------------------------------

First chapter shows the general aspect touched in this paper, starting from the field of the problem and its uses, ending at the dissertations main topic, additionally describing aim and scope of the thesis, and its structure. 

Following chapter guides user through the machine learning topic and usually encountered problems regarding computational complexity and resource usage. Those issues are the base argument of suggesting C++ language as a desired technology for solving them via various libraries.

Data engineering is the main topic of the third chapter. It describes the choice and preparation of biostatistics datasets for the learning process, and selected benchmark methods. The reader is presented with the way of data normalization, selection of regressors, and the control results of selected methods.

Three of the following chapters contain the analysis of selected main functionalities of Shogun, Shark-ML and Dlib libraries respectively. They discuss how does the way of work with each of the products look like, starting from accepted data formats, through data manipulation, ending on available models and their analysis.

Seventh chapter sums up the similarities and differences between mentioned libraries based on results of benchmarking machine learning methods, and available functionalities. Additionally, it describes author's subjective opinion regarding his personal experiences with creating implementations, and working with knowledge sources for each of the product.