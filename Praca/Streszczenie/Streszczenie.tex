
\subsection*{Streszczenie}

    Niniejsza praca ma na celu analizę i porównanie dostępnych w języku C++ bibliotek uczenia maszynowego, pod kątem ich zastosowania w pracy na danych biostatystycznych. W kolejnych rozdziałach czytelnik zapoznawany jest z:
    
    \begin{itemize}
    	\item [$\bullet$] Ogólną postacią problemów napotykanych w procesie implementacji rozwiązań uczenia maszynowego;
    	\item [$\bullet$] Charakterystyką wybranego zestawu danych biostatystycznych wykorzystanych do testów omawianych bibliotek;
    	\item [$\bullet$] Typami oraz uzyskiwanymi wynikami wybranych pod kątem danych eksperymentalnych metod uczenia maszynowego w środowisku prototypowym;
    	\item [$\bullet$] Bibliotekami Tensorflow, Shark, Caffe i PyTorch wraz z metodami implementacji poszczególnych metod wzorcowych;
    	\item [$\bullet$] Zbiorczym podsumowaniem funkcjonalności oferowanych przez wyżej wymienione biblioteki.
    \end{itemize}

	

\vspace{1cm}
\noindent\textbf{Słowa kluczowe:} uczenie maszynowe, C++, biblioteka, sieci neuronowe, głębokie uczenie maszynowe, płytkie uczenie maszynowe.
