
\subsection*{Streszczenie}

    Niniejsza praca ma na celu analizę i porównanie bibliotek uczenia maszynowego dostępnych w języku C++ pod kątem ich zastosowania w pracy na danych biostatystycznych. W kolejnych rozdziałach omawia się:
    
    \begin{itemize}
    	\item [$\bullet$] ogólną postać problemów napotykanych w procesie implementacji rozwiązań uczenia maszynowego;
    	\item [$\bullet$] charakterystykę wybranego zestawu danych biostatystycznych wykorzystanych do testów omawianych bibliotek;
    	\item [$\bullet$] typ oraz uzyskane wyniki wybranych metod uczenia maszynowego pod kątem danych eksperymentalnych w środowisku prototypowym;
    	\item [$\bullet$] biblioteki Shogun, Shark-ML i Dlib wraz z metodami implementacji poszczególnych metod wzorcowych;
    	\item [$\bullet$] zbiorcze podsumowanie funkcjonalności oferowanych przez wyżej wymienione biblioteki.
    \end{itemize}

	

\vspace{1cm}
\noindent\textbf{Słowa kluczowe:} uczenie maszynowe, C++, biblioteka, sieci neuronowe, głębokie uczenie maszynowe, płytkie uczenie maszynowe.
