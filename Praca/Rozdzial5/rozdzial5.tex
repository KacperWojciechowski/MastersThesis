\chapter{Biblioteka Shark-ML}

\section{Wprowadzenie}

Shark-ML to biblioteka uczenia maszynowego dedykowana dla języka C++. Posiada ono otwarte źródło, i udostępniana jest na podstawie licencji \textit{GNU Lesser General Public License}. Głównymi aspektami na których skupia się ta biblioteka są problemy liniowej i nieliniowej optymalizacji (w związku z czym posiada ona część funkcjonalności biblioteki do algebry liniowej), maszyny jądra (np. maszyna wektorów nośnych) i sieci neuronowe. \cite{shark} Podmiotami udostępniającymi bibliotekę jest Uniwersytet Kopenhagi w Danii, oraz Instytut Neuroinformatyki z Ruhr-Universitat Bochum w Niemczech.

\section{Formaty źródeł danych}
\section{Metody przetwarzania i eksploracji danych}
\section{Modele uczenia maszynowego}

Jednym z podstawowych modeli oferowanych przez niniejszą bibliotekę jest regresja liniowa. Do celów jej reprezentacji dostępna jest klasa \textit{LinearModel}, oferująca rozwiązanie problemu w sposób analityczny za pomocą klasy trenera \textit{LinearRegression}, lub podejście iteracyjne implementowane przez klasę trenera \textit{LinearSAGTrainer}, wykorzystujące iteracyjną metodę gradientu średniej statystycznej (ang. \textit{Statistic Averagte Gradient, SAG}).

\section{Metody analizy modeli}

https://www.shark-ml.org/doxygen\texttt{\char`_}pages/html/group\texttt{\char`_}\texttt{\char`_}lossfunctions.html

SquaredLoss

\section{Dostępność dokumentacji i źródeł wiedzy}