\chapter{Shogun}

\section{Introductions}

Shogun is an open-source free machine learning library made in C++, accessible based on \textit{BSD 3-clause} license \cite{shogun:github}. It provides interfaces for various languages, such as Python, Ruby or C\#, but also allows users to use it in its native language. The library focuses on classification and regression problems.

\section{Data formats}

Base class allowing for loading data into Shogun is \textit{std::vector} present in the Standard Template Library (STL) of C++ language. As such, a user can use any mechanism of loading and parsing data, eg. from a file, network or another device, that at the end returns a vector containing the observations, as long as he provides it himself. One popular choice for storing training data is CSV file, for which Shogun provides support \cite{handsOnMachineLearning}. Still in that case data is a subject to several requirements:

\begin{itemize}
	\item \textbf{The file can only contain numeric data} - if any text data are present, preprocessing is needed to recode them into some numeric values (eg. one-hot encoding or regular subsequent numbers in terms of response variable classes). Sadly this means that data explanations cannot be stored alongside it.
	
	\item \textbf{Comma separator} - this becomes an issue if data is previously processed via other software, like Microsoft Excel or JMP, as it commonly uses semicolon as a separator. Despite the CSV format allows for several different separators, Shogun accepts only comma as a valid one.

	\item \textbf{Real values should use dot character as a decimal point} - this comes from the characteristics of C++ (and also other languages), and is required for the language to be able to automatically parse a value from text representation to numeric representation.
\end{itemize}

In order to read and parse data from CSV file, user needs to use \textit{shogun::CCSVFile} class, which result is then loaded into an object of \textit{shogun::SGMatrix}. Because of the fact that this class saved the stored data column-wise, to use them in the training process, user needs to transpose it, and then separate this matrix into two parts, one of which contains predictors, and the other labels. Sample code snippet of this procedure is shown on listing \ref{shogun:csv}. After correct data separation, the container needs to be transposed again so it will be accepted by the learning algorithm, and it needs to be wrapped into specifically designed classes called \textit{CDenseFeatures}, \textit{CMulticlassLabels} or \textit{CRegressionLabels}. This part was shown on listing \ref{shogun:csv2}.

\cppcode{Result/inc/shogun/csv.hpp}{Sample read and preprocessing of data from a CSV file}{shogun:csv}

\cppcode{Result/inc/shogun/shogunModels.hpp}{Repackaging data into desired containers}{shogun:csv2}

\section{Data processing and exploration methods}

\subsection{Normalizing}

The library provides a normalizing min-max mechanism that guarantees the data will be rescaled into a $\langle$ 0; 1 $\rangle$ range, with a class \textit{shogun::CRescaleFeatures}. An object of this class can be then used again for the new data corresponding to the previously learned predictors. Its two main methods are:

\begin{itemize}
	\item \textit{fit()} - teaches the normalizer object statistical characteristics of the provided data;
	\item \textit{transform()} - performs normalizing.
\end{itemize}

In case of some algorithms provided by Shogun, normalizing is one of the first steps executed automatically, so it is not always necessary to do this in preprocessing. Such information should be included in the documentation of a given method. Listing \ref{shogun:normalizer} shows how to use the aforementioned class. The class, aswell as the function shown on the listing perform the normalizing in-place, hence no need to override the object storing the data.

\cppcode{Result/inc/shogun/rescale.hpp}{Przykład funkcji wykonującej normalizację.}{shogun:normalizer}

\subsection{Dimensionality reduction}

Shogun gives to the user several algorithms of dimensionality reduction as following classes \cite{handsOnMachineLearning}:

\begin{itemize}
	\item \textbf{Principle Component Analysis} - class \textit{CPCA};
	\item \textbf{Kernel Principle Component Analysis} - class \textit{CKernelPCA};
	\item \textbf{Multidimensional Scaling} - class \textit{MultidimensionalScaling};
	\item \textbf{IsoMap} - class \textit{CIsoMap};
	\item \textbf{ICA} - class \textit{CFastICA};
	\item \textbf{Factor Analysis} - class \textit{CFactorAnalysis};
	\item \textbf{t-SNE} - class \textit{CTDistributedStochasticNeighborEmbedding}.
\end{itemize}

Each of the classes mentioned above operates by first learning the parameters of the training data by calling the \textit{fit()} function, and establishing the desired dimensions count (except for ICA). Trained reductor object can be used to reduce the dimensionality by \textit{apply\char`_to\char`_feature\char`_vector()} method, that returns transformed data, or in case of ICA, Factor Analysis and t-SNE, by \textit{transform()} method, which result needs to be casted into a pointer to \textit{CDenseFeatures}. Unfortunately, using any of the reductor objects requires creating a new copy of the data object during transformation, instead of performing it in-place. Listing \ref{shogun:reduction} shows an example based on class \textit{CKernelPCA}.

\cppcode{Result/inc/shogun/pca_reduction.hpp}{Dimensionality reduction using Kernel PCA \cite{handsOnMachineLearning}.}{shogun:reduction}

\subsection{L1 and L2 Regularisation}

In Shogun, regularisation is an integral part of a machine learning model, and as such, it is always performed by given model. Each model defines itself which kind of regularization it performs, and this cannot be changed.

\section{Machine Learning Models}
\subsection{Linear Regression}
One of the fundamental machine learning algorithms provided by the Shogun library is linear regression, performed by the \textit{CLinearRidgeRegression} class. As the name suggests, it utilizes Ridge Regression, configured at the model creation stage. Listing \ref{shogun:linear} shows how to adjust the model.

\cppcode{Result/inc/shogun/linear.hpp}{Linear regression example in Shogun}{shogun:linear}

\subsection{Logistic Regression}
Shogun library contains implementation of multiclass logistic regression via \textit{CMulticlassLogisticRegression} class. It also provides configurable regularisation. Listing \ref{shogun:logistic} shows how to use it.

\cppcode{Result/inc/shogun/logistic.hpp}{Logistic regression example in Shogun}{shogun:logistic}

\subsection{Support Vector Machine}
Similarly to logistic regression, Shogun provides the multiclass Support Vector Machine model implementation for classification tasks, as \textit{CMulticlassLibSVM} class. It contains several configurable parameters, along with the choice of kernel itself. Listing \ref{shogun:svm} shows an example of use.

\cppcode{Result/inc/shogun/svm.hpp}{Support Vector Machine example in Shogun}{shogun:svm}

\subsection{K-Nearest Neighbours}

K-Nearest Neighbours algorith is available as \textit{CKNN} class. It allows user to select the method of calculating distances by passing a correct object, and the count of neighbours considered the nearest. The main available distance metrics are: Euklidean, Hamming, Manhattan and cosinus similarity. In comparison to other methods, it does not require the configuration of hyperparameters, allowing the use of it in crossvalidation without problems. Listing \ref{shogun:knn} presents an example configuration and use of KNN algorithm using Euclidean distance.

\cppcode{Result/inc/shogun/knn.hpp}{KNN algorithm example in Shogun}{shogun:knn}

\subsection{Ensemble algorithms}
\subsubsection{Gradient boosting}

Implementation of ensemble algorithm using gradient boosting is adjusted only for regression models. The class \textit{CStochasticGBMachine} is responsible for its execution. It allows the user to configure several parameters, some of which are:

\begin{itemize}
	\item base algorithm;
	\item loss function;
	\item iteration count;
	\item learning coefficient;
	\item fraction of vectors to choose at each iteration.
\end{itemize}

Listing \ref{shogun:gb} shows a method of creating such model using binary decisive tree for regression and classification (implemented by \textit{CCARTree} class) as a base algorithm.

\cppcode{Result/inc/shogun/gb.hpp}{Gradient boosting example in Shogun}{shogun:gb}

\subsubsection{Las losowy}
Metoda lasu losowego jest dostępna w bibliotece Shogun poprzez użycie klasy \textit{CRandomForest}. W przeciwieństwie do wzmacniania gradientu, implementacja tej metody pozwala także na dokonywanie klasyfikacji. Do głównych konfigurowalnych parametrów należą:

\begin{itemize}
	\item liczba drzew;
	\item liczba zbiorów na które powinny zostać podzielone dane;
	\item algorytm wybrania końcowego wyniku;
	\item typ rozwiązywanego problemu;
	\item ciągłość wartości regresorów.
\end{itemize}

Listing \ref{shogun:rf} pokazuje jak utworzyć i skonfigurować model losowego lasu do wykonania zadania aproksymacji funkcji kosinus.

\cppcode{Result/inc/shogun/rforest.hpp}{Przykład użycia metody losowego lasu}{shogun:rf}

\subsection{Sieć neuronowa}

Pierwszym krokiem tworzenia sieci neuronowej dla niniejszej biblioteki jest skonfigurowanie architektury sieci za pomocą obiektu klasy \textit{CNeuralLayers}. Posiada ona szereg metod, które tworzą odpowiednio skonfigurowane warstwy z wybraną funkcją aktywacji:

\begin{itemize}
	\item \textit{input()} - warstwa wejściowa z określoną ilością wymiarów;
	\item \textit{logistic()} - warstwa w pełni połączona z sigmoidalną funkcją aktywacji;
	\item \textit{linear()} - warstwa w pełni połączona z liniową funkcją aktywacji;
	\item \textit{rectified\char`_linear()} - warstwa w pełni połączona z funkcją aktywacji ReLU;
	\item \textit{leaky\char`_rectified\char`_linear} - warstwa w pełni połączona z funkcją aktywacji Leaky ReLU;
	\item \textit{softmax} - warstwa w pełni połączona z funkcją aktywacji softmax. 
\end{itemize}

Kolejność wywoływania powyższych metod jest istotna, ponieważ decyduje ona o kolejności warstw w modelu. Po zakończeniu konfiguracji, możliwe jest utworzenie obiektu zatwierdzonej architektury za pomocą funkcji \textit{done()}, a następnie wykorzystanie go do inicjalizacji klasy \textit{CNeuralNetwork}. W celu połączenia warstw, należy wywołać na obiekcie sieci neuronowej funkcję \textit{quick\char`_connect} oraz zainicjalizować wagi metodą \textit{initialize\char`_neural\char`_network}. Może ona przyjąć parametr określający rozkład Gaussa używany do inicjalizacji parametrów.

Następnym krokiem jest skonfigurowanie optymalizatora za pomocą metody \textit{set-\char`_optimization}. Klasa \textit{CNeuralNetwork} wspiera optymalizację z wykorzystaniem metody najszybszego spadku oraz Broydena-Fletchera-Goldfarba-Shannona. Sieć neuronowa posiada wbudowaną regularyzację L2, którą można skonfigurować, podobnie jak pozostałe parametry takie jak współczynnik uczenia, liczba epok, kryterium zbieżności dla funkcji straty, czy wielkość zestawów wsadowych. Niestety, niemożliwy jest wybór funkcji straty, gdyż jest on dokonywany automatycznie na podstawie typu zmiennej odpowiedzi. Listing \ref{shogun:nn} przedstawia pełny proces budowania, konfiguracji oraz uczenia sieci. Niestety ze względu na brak implementacji warstwy neuronu o aktywacji w postaci tangensa hiperbolicznego, na potrzeby prezentacji zastosowania zdecydowano się na wykorzystanie funkcji reLU, co ma wpływ na uzyskane wyniki.

\cppcode{Result/inc/shogun/neural.hpp}{Przykład użycia sieci neuronowej.}{shogun:nn}

\section{Metody analizy modeli}

\subsection{Błąd średniokwadratowy}

Obliczenie błędu średniokwadratowego w bibliotece Shogun sprowadza się do utworzenia obiektu wykorzystującego typ \textit{CMeanSquaredError} jako argument szablonu funkcji \textit{some<>()}. Jest on zwracany pod postacią wskaźnika. W celu otrzymania wartości błędu dla posiadanych danych, należy wywołać z jego pomocą funkcję \textit{evaluate}, do której przekazany zostaje zestaw predykcji modelu oraz zaobserwowanych wartości odpowiedzi. Listing \ref{shogun:mse} ukazuje sposób użycia wspomnianego mechanizmu.

\cppcode{Rozdzial4/shogun-mse.cpp}{Przykład obliczenia wartości błędu średniokwadratowego \cite{handsOnMachineLearning}}{shogun:mse}

\subsection{Średni błąd bezwzględny}

Realizacja obliczania średniego błędu bezwzględnego dla biblioteki Shogun dokonywana jest za pomocą klasy \textit{CMeanAbsoluteError} pełniącej rolę ewaluatora. Tworzona jest ona poprzez wykorzystanie szablonu \textit{some<>} a następnie wykorzystywana do obliczeń wołając jej metodę \textit{evaluate} przekazując uzyskane oraz oczekiwane wyniki regresji lub klasyfikacji. Listing \ref{shogun:mae} przedstawia sposób użycia wyżej wymienionej klasy.

\cppcode{Rozdzial4/shogun-mae.cpp}{Przykład obliczenia wartości średniego błędu bezwzględnego \cite{handsOnMachineLearning}.}{shogun:mae}

\subsection{Logarytmiczna funkcja straty}

Logarytmiczna funkcja straty jest możliwa do obliczenia z wykorzystaniem biblioteki Shogun przy użyciu klasy \textit{CLogLoss}, jednak udostępniane przez nią metody wskazują że powinna być wykorzystywana przez model, a nie bezpośrednio przez użytkownika. Udostępnia ona metodę \textit{get\char`_square\char`_grad()} pozwalającą na obliczenie kwadratu gradientu między zadaną predykcją a docelowym wynikiem. Sposób użycia tej metody zaprezentowano na listingu \ref{shogun:log}

\cppcode{Rozdzial4/shogun-log.cpp}{Przykład użycia klasy \textit{CLogLoss}.}{shogun:log}

\subsection{Metryka $R^2$}

Biblioteka Shogun nie posiada bezpośredniej implementacji dla metryki $R^2$ w związku z czym, pomimo możliwości wykorzystania wbudowanej metody obliczania błędu średniokwadratowego, wariancja odpowiedzi potrzebna dla uzyskania wyniku musi zostać uzyskana przez własny mechanizm użytkownika. Listing \ref{shogun:verify} przedstawia funkcję weryfikującą poprawność modeli opisanych w poprzednich punktach, obliczającą wartość metryki $R^2$.

\cppcode{Result/inc/shogun/verify.hpp}{Przykład obliczenia metryki $R^2$.}{shogun:verify}

\subsection{Dokładność}

Do obliczenia dokładności w przypadku zadań regresji, biblioteka udostępnia klasę \textit{CMulticlassAccuracy}. Pozwala ona nie tylko na samą klasyfikację, lecz także oferuje metodę pobrania macierzy błędnych klasyfikacji. Nie znaleziono natomiast klasy \textit{CAccuracyMeasure} wspomnianej w pracy \cite{handsOnMachineLearning}, co sugerowałoby jej usunięcie z biblioteki. Listing \ref{shogun:acc} pokazuje w jaki sposób należy użyć klasy \textit{CMulticlassAccuracy}.

\cppcode{Rozdzial4/shogun-acc.cpp}{Przykład obliczenia dokładności modelu.}{shogun:acc} 

\subsection{Precyzja i pełność (recall), oraz metryka F1}

W podręczniku \cite{handsOnMachineLearning} wspomniane zostały klasy \textit{CRecallMeasure} oraz \textit{CF1Measure} mające pozwolić obliczyć odpowiednio pamięć modelu oraz metrykę F1, jednak w trakcie pracy z biblioteką nie znaleziono definicji tych klas, lub jakichkolwiek innych pełniących te funkcje, w związku z czym założono brak implementacji tych metod dla biblioteki Shogun. Zdecydowano się na wspomnienie o tym fakcie, w celu podkreślenia możliwoścy wystąpienia rozbieżności między źródłami wiedzy, a aktualnym stanem biblioteki.

\subsection{Pole pod wykresem krzywej operacyjnej}

Biblioteka Shogun posiada implementację obliczania pola pod wykresem krzywej charakterystycznej odbiornika, w postaci klasy \textit{CROCEvaluation}. Listing \ref{shogun:roc} przedstawia sposób jej użycia.

\cppcode{Rozdzial4/shogun-roc.cpp}{Przykład obliczenia pola pod wykresem funkcji ROC dla Shogun.}{shogun:roc}

\subsection{K-krotny sprawdzian krzyżowy}

Sprawdzian krzyżowy stanowi w bibliotece Shogun złożony mechanizm, do którego wykorzystania należy przygotować drzewo decyzyjne parametrów, reprezentowane przez klasę \textit{CModelSelectionParameters}. Użytkownik może wybrać model oraz kryterium ewaluacji modelu poprzez utworzenie odpowiednich klas, a następnie przekazanie ich w konstruktorze obiektu sprawdzianu krzyżowego, będącego instancją klasy \textit{CCrossValidation}. Kolejnym krokiem jest utworzenie instancji klasy \textit{CGridSearchModelSelection} która dokona wyboru parametrów. Ostatnim etapem jest konfiguracja docelowego modelu i przeprowadzenie procesu uczenia. Dokłady wygląd całego mechanizmu został przedstawiony na listingu \ref{shogun:cross}.

\cppcode{Result/inc/shogun/cross.hpp}{Przygotowanie modelu wieloklasowej regresji liniowej z wykorzystaniem sprawdzianu krzyżowego.}{shogun:cross}

\section{Dostępność dokumentacji i źródeł wiedzy}

Internetowe źródła informacji w postaci forów społecznościowych skupiają się na wykorzystaniu biblioteki Shark w innych językach, jak np. Python, lecz wraz z jej kodem źródłowym na platformie GitHub \cite{shogun:github} możliwe jest wygenerowanie przykładów jej wykorzystania także w języku C++ w folderze \textit{examples}. Przykłady te należy zbudować za pomocą odpowiednego skryptu Pythona zawartego w repozytorium, powodując wygenerowanie listingów kodów w docelowym języku w plikach JSON. Niestety, okazują się one obrazować użycie biblioteki w nienaturalny, proceduralnie generowany sposób, sprawiając, że przy faktycznej próbie skorzystania z API projektu, stają się one bezużyteczne. Dodatkowo, jedyna forma dokumentacji projektu ogranicza się do komentarzy w kodzie, zmuszając użytkownika do błądzenia po repozytorum w poszukiwaniu potrzebnych informacji. 

Shogun jest jedną z bibliotek opisaną w podręczniku \cite{handsOnMachineLearning}, wprowadzającym czytelnika zarówno do podstawowych funkcjonalności Shogun, jak i podsumowującej podstawy teorii uczenia maszynowego w kontekście ich zastosowania. Większość z przykładów realizacji poszczególnych typów modeli w tej książce posiada przedstawione główne fragmenty listingów dla biblioteki Shogun. Warto jednak zaznaczyć, że różnią się one od przykładów generowanych przez skrypt budujący, obrazując wykorzystanie faktycznie udostępnianego API biblioteki. W toku pracy nad niniejszym rozdziałem, książka ta okazała się jedynym wartościowym źródłem wiedzy na jego temat.