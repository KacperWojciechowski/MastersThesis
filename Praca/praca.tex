%==========================================================================================
%
%								Uniwersytet Zielonogórski
%
%						SZABLON PRACY DYPLOMOWEJ W PAKIECIE LaTeX
%							wykonany podczas zajęć seminaryjnych
%				pod przewodnictwem prof. dr hab. inż Dariusza Ucińskiego
%
%							 Damian Kowalów, Mariusz Buciakowski
%
% 				       Wydział Informatyki, Elektrotechniki i Automatyki
%						Instytut Sterowania i~Systemów Informatycznych
%
%							 Zielona Góra, kwiecień 2013
%			   (ostatnia modyfikacja: 15.11.2019 przez Paweł Jamroziak)
%==========================================================================================
%\documentclass[a4paper,12pt]{book}
\documentclass[a4paper,12pt,openany]{book}
% ------------------------------------------------------------------------
% pakiet do wzorów ams
% ------------------------------------------------------------------------
\usepackage{amsmath}
\usepackage{amssymb}

% ------------------------------------------------------------------------
% język polski
% ------------------------------------------------------------------------
\usepackage[polish]{babel}
\usepackage{polski}
\usepackage[utf8]{inputenc}
\usepackage[T1]{fontenc}

% ------------------------------------------------------------------------
% obsługa pdf
% ------------------------------------------------------------------------
\usepackage[pdftex,usenames,dvipsnames]{color}	%obsługa kolorów

% ------------------------------------------------------------------------
% wstawienie danych o autorze i pracy 
% ------------------------------------------------------------------------
\usepackage[pdftex,
                hidelinks=true,
				pagebackref=false,						% referencje w spisie literatury do strony na, której została użyta
				draft=false,								% draft
				pdfpagelabels=false,						%
				pdfstartview=FitV,						% lub FitH
				pdfstartpage=1,							% 
				bookmarks=true,							% zakładki w pliku pdf
				pdfauthor={Kacper Wojciechowski},						% należy wpisać autora pracy
				pdftitle={Praca inżynierska},			% 
				pdfsubject={Metody realizacji interfejsu sieciowego w modułach IoT},				% tytuł
				pdfkeywords={interfejs sieciowy, mikrokontroler, system wbudowany, iot},			% słowa kluczowe
				unicode=true]{hyperref}   


% ------------------------------------------------------------------------
%	style
% ------------------------------------------------------------------------
\usepackage{extsizes}							%więcej rozmiarów czcionek
\usepackage[a4paper,left=3.5cm,right=2.5cm,top=2.5cm,bottom=2.5cm]{geometry}
\usepackage{tocloft}								% format spisu treści
\usepackage{array}								% lepiej wyglądające tabelki
\usepackage[format=hang,
				labelsep=period,
				labelfont={bf,small},
				textfont=small]{caption}		% formatuje podpisy pod rysunkami i tabelami
\usepackage{floatflt}							% ładniejsze opisywanie obrazków tekstem
\usepackage{subfig}								% możliwość wstawiania figur w kolumnach
\usepackage{graphicx}							% do obsługi grafiki
\usepackage{here}									% wymuszanie położenia figury w danym miejscu
\usepackage{url}									% adresy internetowe
\usepackage{enumerate}							% modyfikowanie list wyliczeniowych np \begin{enumerate}[(a)]...
\usepackage{multirow}							% do tabel 
% ------------------------------------------------------------------------
% listingi
% ------------------------------------------------------------------------
\usepackage{listings}							% do wstawiania listingow programów




\usepackage{slantsc} % Pochyłe kapitaliki  np. \textsl{\textsc{Automatyka i robotyka}}

% ------------------------------------------------------------------------
% inne
% ------------------------------------------------------------------------
\usepackage{glossaries}

\usepackage{dashrule}
\usepackage{fancyhdr} 							% do stopki i nagłówka
\usepackage{calc}
\usepackage{packages/zmienne}					% zmienne dodatkowe używane min. w karcie pracy oświadczeniu i stronie tytułowej zebrane w jednym miejscu
\usepackage{packages/strona_tytulowa}
\usepackage{packages/oswiadczenie}
\usepackage{packages/karta_pracy}
\usepackage{packages/pusta_strona}
\usepackage{packages/wspolrealizacja}
\usepackage{longtable}							% do podziału tabel na wiele stron

\usepackage{listingsutf8}
\usepackage{longtable}

\lstset{
literate={ą}{{\k{a}}}1 {Ą}{{\k{A}}}1 {ć}{{\'c}}1 {Ć}{{\'{C}}}1 {ę}{{\k{e}}}1 {Ę}{{\k{E}}}1 {ł}{{\l{}}}1 {Ł}{{\L{}}}1 {ń}{{\'n}}1 {Ń}{{\'N}}1 {ó}{{\'o}}1 {Ó}{{\'O}}1 {ś}{{\'s}}1 {Ś}{{\'S}}1 {ż}{{\.z}}1 {Ż}{{\.Z}}1 {ź}{{\'z}}1 {Ź}{{\'Z}}1
}

% ------------------------------------------------------------------------
% kodowanie czcionek
% ------------------------------------------------------------------------
\usepackage[T1]{fontenc}
\usepackage{lmodern}\normalfont %to load T1lmr.fd 

% ------------------------------------------------------------------------
% do algorytmów
% ------------------------------------------------------------------------

\usepackage{algorithm}
\usepackage{algorithmic}
\floatname{algorithm}{Algorytm}

% ------------------------------------------------------------------------
% do nomenklatury
% ------------------------------------------------------------------------

\usepackage[section]{placeins}
\usepackage{nomencl}
\makenomenclature
\usepackage{makeidx}
\makeindex
\renewcommand{\nomname}{Spis ważniejszych symboli}
% na końcu pliku z nomenklaturą należy umieścić polecenie
% \printnomenclature
% plik należy dodać poleceniem \input
%% przykład treści pliku
%% \nomenclature{$\oplus$}{Dylatacja zbioru}
%% \printnomenclature

% ------------------------------------------------------------------------
% do bibliografii
% ------------------------------------------------------------------------
\usepackage[backend=bibtex, style=numeric, sorting=none]{biblatex}
%\addbibresource{Bibliografia/bibliografia.bib}
%\bibliographystyle{unsrt}

% ------------------------------------------------------------------------
%   Kropki po numerach sekcji, podsekcji, itd.
%   Np. 1.2. Tytuł podrozdziału
% ------------------------------------------------------------------------
\makeatletter
    \def\numberline#1{\hb@xt@\@tempdima{#1.\hfil}}                      %kropki w spisie treści
    \renewcommand*\@seccntformat[1]{\csname the#1\endcsname.\enspace}   %kropki w treści dokumentu
\makeatother

\makeatother
% ------------------------------------------------------------------------
% Definicje
% ------------------------------------------------------------------------
\def\nonumsection#1{%
    \section*{#1}%
    \addcontentsline{toc}{section}{#1}%
    }
\def\nonumsubsection#1{%
    \subsection*{#1}%
    \addcontentsline{toc}{subsection}{#1}%
    }
\reversemarginpar %umieszcza notki po lewej stronie, czyli tam gdzie jest więcej miejsca
\def\notka#1{%
    \marginpar{\footnotesize{#1}}%
    }
%\def\mathcal#1{%
%    \mathscr{#1}%
%    }

\newcommand{\myemptypage}{ \newpage  \thispagestyle{empty}~\newpage}
\usepackage{makecell}

%-------------------------------------------------------------------------
% listingi
%-------------------------------------------------------------------------
\definecolor{commentColor}{rgb}{0,0.643,0}
\definecolor{keywordColor}{rgb}{0,0,0.835}
\lstdefinestyle{praca}{basicstyle=\footnotesize\ttfamily,
                        keywordstyle=\color{keywordColor},
                        commentstyle=\color{commentColor},
                        numbers=left,
                        stepnumber=1,
                        numberstyle=\scriptsize,
                        numbersep=10pt,
                        basewidth=0.5em,
                        extendedchars=true,
                        frame=tb}

\newcommand{\cppcode}[3]{\vspace{8pt}\lstinputlisting[caption=#2, style=praca, language=C++, label=#3, xleftmargin=.02\textwidth, xrightmargin=.02\textwidth]{#1}}

%-------------------------------------------------------------------------
% stopka i nagłówek
%-------------------------------------------------------------------------
\setlength{\headheight}{15pt}

\pagestyle{fancy}
\renewcommand{\chaptermark}[1]{\markboth{#1}{}}
\renewcommand{\sectionmark}[1]{\markright{#1}{}}

\fancyhf{}
\fancyhead[LE,RO]{\thepage}
\fancyhead[RE]{\textit{\nouppercase{\leftmark}}}
\fancyhead[LO]{\textit{\nouppercase{\rightmark}}}

\fancypagestyle{plain}{ %
\fancyhf{}
\renewcommand{\headrulewidth}{0pt}
\renewcommand{\footrulewidth}{0pt}}

% ------------------------------------------------------------------------
% Inne
% ------------------------------------------------------------------------
\frenchspacing
\setlength{\parskip}{3pt}           	%odstęp pomiędzy akapitami
%\linespread{1.49}                    	%odstęp pomiędzy liniami (interlinia)
\setcounter{tocdepth}{3}
\setcounter{secnumdepth}{3}


% ------------------------------------------------------------------------
% Polskie podpisy
% ------------------------------------------------------------------------
\renewcommand{\figurename}{Figure}
\renewcommand{\tablename}{Table}

% ------------------------------------------------------------------------
% Bibliografia
% ------------------------------------------------------------------------
%\bibliographystyle{unsrt}					% kolejność według użycia
%\bibliographystyle{plain}					% kolejność alfabetyczna
\bibliography{Bibliografia/bibliografia.bib}
  
  

%==========================================================================================
% Deklaracja fontow kapitalikowych z kodowaniem T1
%==========================================================================================
\DeclareFontShape{T1}{lmr}{bx}{sc} { <-> ssub * cmr/bx/sc }{}
\DeclareFontShape{T1}{lmr}{bx}{scit}{<-> ssub * cmr/bx/scsl}{}
%==========================================================================================
% Inne deklaracje
%==========================================================================================



%==========================================================================================
% Dane na temat pracy do wypełnienia
%==========================================================================================
\author{Kacper Wojciechowski}
\kierunek{Informatyka}
\grupa{21 INF-ZSI-SD}		% np. 101AiRDz
\title{Analiza porównawcza bibliotek uczenia maszynowego języka C++ na potrzeby zastosowań w biostatystyce}
\tytulAngielski{Comparative analysis of machine learning libraries in C++ for applications in biostatistics}
\uczelnia{Uniwersytet Zielonogórski}
\wydzial{Wydział Informatyki, Elektrotechniki i Automatyki}
\praca{Praca dyplomowa} % pozostawić właściwe
\promotor{Prof. dr hab. inż. Dariusz Uciński}
\konsultant{} 	% nie usuwać w~przypadku braku lecz pozostawić puste czyli: \konsultant{}
%\konsultant{}
\miasto{Zielona Góra}
\miesiac{czerwiec}	% miesiąc złożenia pracy
\rok{2023}
\dzien{dd} 			% dzień podpisania oświadczenia (cyfrowo) np. 10
\mm{mm} 				% miesiąc podpisania oświadczenia (cyfrowo) np. 01 (styczeń)

%==========================================================================================
% Uwaga! polecenie \myemptypage dodaje pustą stronę do treści. Praca oddana do dziekanatu powinna być zbudowana zgodnie z szablonem znajdującym się na stronie WEIT (z jedną stroną pustą następującą po karcie pracy), jednak jeżeli student planuje wydruk dla siebie, zalecane jest zastąpienie poleceń \newpage poleceniem \myemptypage. Skutkuje to utworzeniem bardziej przejrzystego układu zgodnego z zaleceniami pisania książek, czyli rozpoczynanie nowych treści na prawej stronie.
%==========================================================================================


%==========================================================================================
% Dokument glowny
%==========================================================================================
\begin{document}
\pagenumbering{roman}
%=========================================================================================
% Strona tytułowa
%=========================================================================================
\thispagestyle{empty}
\stronatytulowa

\thispagestyle{empty}
\pustastrona
\newpage
%=========================================================================================
% Oświadczenie o~współrealizacji pracy - w~przypadku braku zakomentować lub usunąć sekcję
%========================================================================================
%\autorDwa{Imię i~nazwisko współautora}
%\wspolrealizacjaTrescJeden{wykaz czynności wykonanych przez tę osobę}
%\wspolrealizacjaTrescDwa{wykaz czynności wykonanych przez tę osobę}
%\wspolrealizacja
%\newpage

\thispagestyle{empty}
\newpage

%========================================================================================
% Streszczenie
%========================================================================================
\normalsize

\subsection*{Streszczenie}

	This dissertation aims to analyze and compare machine learning libraries available for C++ for use in biostatistics. The following chapters describe:

    \begin{itemize}
    	\item [$\bullet$] general problems encountered in the process of implementing machine learning solutions;
    	\item [$\bullet$] characteristics of biostatistics datasets used for testing chosen libraries;
    	\item [$\bullet$] control results of the methods selected for libraries comparison purpose;
    	\item [$\bullet$] Shogun, Shark-ML and Dlib libraries with implementations of selected machine learning methods;
    	\item [$\bullet$] summary of featurs offered by each library in respect to each other.
    \end{itemize}

\vspace{1cm}
\noindent\textbf{Keywords:} machine learning, C++, library, neural network, deep machine learning, shallow machine learning.

\newpage
%\myemptypage

%========================================================================================
% Spis tresci, spis tabel i~rysunków
%========================================================================================
%spis tresci
	\tableofcontents
	\newpage
	%\myemptypage
%spis rysunków
 	\listoffigures
	\newpage
	%\myemptypage
%spis tabel
	\listoftables
	\newpage
	%\myemptypage

%========================================================================================
% Licznik Stron
%========================================================================================
\newcounter{licznikStron}
\setcounter{licznikStron}{\value{page}}
\setcounter{licznikStron}{1}
\pagenumbering{arabic}
\setcounter{page}{\value{licznikStron}}

%========================================================================================
% Tresc
%========================================================================================
\chapter{Wstęp}
\section{Wprowadzenie} % -------------------------------------------------------------------------

We współczesnym stanie techniki coraz częściej można spotkać się z urządzeniami i programami o inteligentnych funkcjach, takich jak predykcja zjawisk na podstawie zestawu danych, rozpoznawanie obrazu, analiza mowy, czy przetwarzanie języka naturalnego. Znajdują one zastosowanie w różnych dziedzinach codziennego życia, m.in. w medycynie. W zależności od potrzeb, techniki uczenia maszynowego można wykorzystać do zastosowań medycznych, jak np. rozpoznawanie komórek rakowych na skanach rezonansem magnetycznym, podejmowanie decyzji na podstawie zbioru objawów obecnych u pacjenta, lub przewidywanie norm związków naturalnie występujących w organiźmie ludzkim w zależności od okoliczności i wyników pomiarów. 

Jedną z istotnych dziedzin medycyny jest biostatystyka, polegająca na wykorzystaniu analizy statystycznej do wnioskowania na podstawie zbiorów danych, takich jak rezultaty przeprowadzonych badań (np. morfologicznych, poziomu poszczególnych hormonów we krwii, itp.), informacji o nawykach żywieniowych oraz stylu życia pacjenta. Szczególnie istotną formą systemów operujących w tej dziedzinie są systemy eksperckie, wykorzystujące techniki płytkiego i głębokiego uczenia maszynowego w celu wspierania diagnozy stawianej przez wykwalifikowanych lekarzy. 

U podstaw wyżej wymienionych zagadnień leży implementacja rozwiązań opartych o teorię uczenia maszynowego, oraz wszelkie związane z tym problemy. W związku z tym na przestrzeni lat powstało wiele gotowych narzędzi, takich jak biblioteki i \textit{frameworki}, mające na celu wsparcie programistów w szybkim i prawidłowym wprowadzaniu rozwiązań sztucznej inteligencji na różne platformy docelowe oraz w różnych językach, począwszy od języka C++, przez Python, po środowiska takie jak Matlab. 

Istotnym krokiem w przygotowywaniu oprogramowania wykorzystującego sztuczną inteligencję jest prawidłowy wybór wspomnianych wcześniej narzędzi dokonywany na etapie projektowania, tak, aby oferowały one możliwości adekwatne do wymagań funkcjonalnych. Niniejsza praca dokonuje analizy porównawczej bibliotek uczenia maszynowego dla języka C++ w kontekście zastosowań w dziedzinie biostatystyki, celem umożliwienia czytelnikowi trafnego wyboru odpowiedniego narzędzia do realizacji projektu badawczego. Warto zaznaczyć, że niniejsza praca przedstawia jedynie wybrany zakres głównych funkcjonalności omawianych bibliotek ze względu na zastosowania w biostatystyce, w związku z czym mogą one posiadać większą ilość bardziej szczegółowych funkcjonalności, lub nowe metody dodane po utworzeniu niniejszej pracy.

\section{Cel i zakres pracy} % ----------------------------------------------------------------------

Celem pracy jest przeprowadzenie analizy i przygotwanie zestawienia bibliotek do uczenia maszynowego dla języka C++, obrazując przykłady bazujące na zestawie danych biostatystycznych.

Zakres pracy obejmował:

\begin{itemize}
    \item [$\bullet$] Przegląd dostępnych bibliotek języka C++;
    \item [$\bullet$] Inżynierię i kształtowanie danych;
    \item [$\bullet$] Płytkie i głębokie uczenie nadzorowane;
    \item [$\bullet$] Kwestie wydajnościowe w dopasowywaniu i wdrażaniu modeli;
    \item [$\bullet$] Badania praktyczne w oparciu o zestaw danych medycznych i biologicznych.
\end{itemize}

\section{Struktura pracy} % ----------------------------------------------------------------------

Pierwszy rozdział przedstawia ogólnym zagadnieniem dotykanym przez pracę, począwszy od dziedziny problemu i jej zastosowań, do istoty tematu pracy. Dodatkowo omawiany jest cel i zakres realizacji pracy, oraz jej strukturą.

Kolejny rozdział wprowadza czytelnika do tematu uczenia maszynowego, oraz napotykanych w nim problemów dotyczących złożoności obliczeniowej oraz zużycia zasobów. Stanowią one podstawę do zaproponowania języka C++ jako technologi wspierającej ich rozwiązanie przy pomocy bibliotek. 

Tematem rozdziału trzeciego jest przygotowanie elementów testowych do wykorzystania w późniejszej analizie porównawczej. Składa się na nie wybranie i przygotowanie do zestawu danych biostatystycznych do procesu uczenia oraz wybrane wzorcowych rozwiązań. Czytelnik przeprowadzony jest przez normalizację danych i selekcję najlepiej dopasowanych regresorów, oraz zostaje zapoznany z przykładowymi wynikami rozwiązań wzorcowych.

Kolejne trzy rozdziały skupiają się na analizie głównych funkcjonalności wybranych bibliotek pod kątem zastosowań w biostatystyce. Czwarty rozdział przedstawia bibliotekę Shogun, piąty zapoznaje użytkownika z biblioteką Shark-ML, natomiast szósty omawia bibliotekę Dlib. Wprowadzają one czytelnika kolejno w poszczególne aspekty pracy z wybranym produktem, od akceptowanych formatów danych, przez manipulację obserwacjami, po dostępne modele i metody ich analizy. 

Rozdział siódmy zestawia podobieństwa i różnice między bibliotekami  na podstawie wyników przeprowadzonych procesów uczenia, zestawiając wyniki uzyskanych modeli oraz dostępne funkcjonalności w formie tabel. Dodatkowo, zawarta tu została także subiektywna opinia autora w postaci opisów słownych na podstawie jego doświadczeń z implementacją rozwiązań i pracą ze źródłami wiedzy.
\chapter{A practical take at machine learning}
\section{Modern problems}

Machine learning is based on advanced algorithms and complex data structures, performing calculations on training and validation datasets provided by the user, aswell as data received during their regular operation.

Some of the most basic models are created via techniques such as linear and non-linear regression, logistics regression or linear discriminant analysis. They result in shallow machine learning models, which have relatively small computational complexity to evaluate results of a dataset in comparison to other models \cite{shallow}.

Few of the more soffisticated methods are for example decision trees, based on algorythmical tree logic \cite{tree}. Each of the tree layer corresponds to the best predictor available at a current state, causing branching to specific values or ranges of values. The result calculation is achieved via traversing the tree from the root to one of its ending leafs. 

The most advanced, yet at the same time most demanding in complexity and memory types of deep learning models such as deep neural networks, convolutional neural networks and recursive neural networks. They utilize structure consisting of one input layer, one or more hidden layers containing mathematical neurons, and one output layer. Each subsequent layer is connected to the previous one, either fully or partially, feeding the results of processing forward. Basic neural networks make neurons present in the same layer independent from one another. Each connection has its own weight, which is multiplied by the input from that connection, and sumed up with the results from other connections, to be passed forward to the activation function. This function decide whether a neuron should activate, and (in case of continuous range $\langle$0; 1$\rangle$) to what extent \cite{mit_neural}. An example of a network with a singular neuron was presented on figure \ref{fig:nn}.

\begin{figure}[!ht]
    \centering
    \includegraphics[width=150mm]{Rysunki/Rozdzial2/neuron.png}
    \caption{Neuron blueprint \dywiz{} Simplelearn.}
    \label{fig:nn}
\end{figure}

Extensive neural networks, such as CNN, require additional steps to preprocess the input data, such that it is acceptable by the network, like for example pooling. By analizing structures incorporated by each of the mentioned methods, following implementation challenges can be identified \cite{constrained}:

\begin{itemize}
    \item [$\bullet$] Efficiency - it is tightly connected to the computational complexity of the used methods, aswell as the characteristics of used programming language and hardware platform. The desired effect would be to minimize the learning time of the model and propagation time from inputing the data to receiving a result. In most cases, minimalization of propagation time takes priority.
    
    \item [$\bullet$] Memory usage - this is a concern mainly when using platform with limited resources, such as microcontrolers and microcomputers, where current RAM and flash memory sizes (especially in embedded systems) can be very limited in comparison to regular computers or mobile platforms.
\end{itemize}

During development of machine learning technology, significant steps were taken into solving above mentioned problems, and fulfill everincreasing requirements of modern machine learning applications. Some of the ways this was achieved are algorithm optimization, selecting high frequency hardware platforms, utilizing paralel computing and using high performance compiled programming languages, especially those that support low level operations. 

\section{C++ as a solution to machine learning problems}

Among various languages and environments supporting machine learning, such as Python, C++, Java and Matlab, one of the special few is C++. It is an imperative language with strong typing, connecting low level functionalities for specific hardware architectures with high level programming. As such, it offers vast control over memory usage and optimization possibilities, like adjusting used types for the specific processing needs, contol over variable placing in memory (it is up to the developer to choose whether the data will be placed on stack or heap) and function call optimizations by inlining and tail recursion optimization. In contrary to scripting languages whose code is interpreted during execution, such as Python and Matlab language, C++ is a compiled language, which means that its code is converted into a binary executable adjusted to specific CPU architecture. This completely removes the overhead of interpreting the source code, as the conversion to CPU language is done only once, during the binary generation process. Additionally, this allows the compiler to perform various low level optimizations \cite{cpp_char}.

Parts of C++ mechanisms which find their root back in C language allow to use Assembly insets, further increasing the efficiency, at the cost of portability. Some platforms also offer API for hardware acceleration modules, such as eg. Neural Networks API - NNA - of Android, allowing for faster processing via specially designed hardware \cite{android_nna}.

\begin{figure}[!ht]
    \centering
    \includegraphics[width=150mm]{Rysunki/Rozdzial2/multithreading.jpg}
    \caption{Concurrency evolution in modern C++ - Modernes C++.}
    \label{fig:cpp_history}
\end{figure}

One of the major factors increasing efficiency of machine learning models is paralel processing. Availability of multithreading (added in C++11 standard and further developed, as presented in figure \ref{fig:cpp_history}) and compatibility with CUDA language and API \cite{cpp_cuda} allows C++ to perform many calculations simultaneously utilizing multiple cores of the CPU or delegating processing to one or many graphics cards (where the number of GPU processors vastly outnumbers CPU cores). Additional, although fairly obvious benefit of using this language is easy integration of the models with programs already made in the same language (which is often the case in embedded systems, IoT and computationally intensive software).

\section{The aim of creating libraries}

Due to the complexity of mechanisms that are part of machine learning, various experience of developers and the need for intense optimization, implementation of the machine learning methods is lengthy and expensive. Here programmers can seek for help in libraries created by corporations such as Google and big open-source communities. The libraries provide ready for use mechanisms (often created according to object oriented paradigm, via a set of classes), which are constantly updated and optimized by developers who use it in their daily work or passion projects. They offer a way to quickly create your own models, and often also allow to use already created pre-trained models on stock datasets. Important benefit of using existing libraries that are still supported is better stability, as parts of the library are implemented and tested by experienced developers, like in the case of TensorFlow library provided by Google.

Majority of libraries for machine learning, even in languages such as Python, are in fact made in C++, providing API for different languages. Sadly, not all libraries made in C++ offer suitable API to use in that same language, as a result, in most cases its use is convoluted and unnecessarily difficult. As a result, some of the libraries work with models created (even by this very same library) in different language, as is the case with TensorFlow lite. Common example is using Python to create a model graph or exporting model to ONNX format (Open Neural Network Exchange) \cite{cpp_onnx}. In this analysis, presented libraries offer the ability to create models in C++, without the need for other language. 
\chapter{Inżynieria danych eksperymentalnych i testowe szablony modeli}
\section{Omówienie danych eksperymentalnych}
	
	W celu zestawienia funkcjonalnego bibliotek uczenia maszynowego w języku C++ i przedstawienia przykładów konieczne było wybranie danych eksperymentalnych możliwych do wykorzystania jako porównawczy punkt odniesienia. Jako w/w dane wybrano bazę dotyczącą diagnostyki raka piersi, w której zamieszczono wyniki obrazowania określone w sposób liczbowy. Dane mają następującą strukturę:
	
	\begin{enumerate}
		\item [1)] ID - numer identyfikacyjny pacjentki;
		\item [2)] Diagnosis [\textit{Malignant - M} / \textit{Benign - B}] - charakter nowotworu, \textbf{zmienna odpowiedzi};
		\item [3)] Dane klasyfikujące:
			\begin{enumerate}
				\item [a)] \textit{Radius} - średnica guza;
				\item [b)] \textit{Texture} - tekstura guza;
				\item [c)] \textit{Perimeter} - obwód guza;
				\item [d)] \textit{Area} - pole guza;
				\item [e)] \textit{Smoothness} - gładkość, miara lokalnych różnic w promieniu guza;
				\item [f)] \textit{Compactness} - zwartość, wykorzystywana do oceny stadium guza;
				\item [g)] \textit{Concavity} - stopień wklęsłości miejsc guza;
				\item [h)] \textit{Concave points} - punkty wklęsłości guza;
				\item [i)] \textit{Symmetry} - symetria guza, pomagająca w ocenie charakteru przyrostu guza.
				\item [j)] \textit{Fractal dimention (,,coastline approximation'' - 1)} - wymiar fraktalny pozwalający na ilościowy opis złożoności komórek nerwowych, umożliwiający stwierdzenie nowotworzenia się zbioru komórek.
			\end{enumerate}
	\end{enumerate}
	
	Dla każdej ze zmiennych odpowiedzi została zebrana średnia wartość, odcyhelenie standardowe oraz średnia trzech największych pomiarów, gdzie każdy zestaw ustawiony jest sekwencyjnie (np. kolumna 3 - średni promień, kolumna 12 - odchylenie standardowe promienia, kolumna 22 - średnia trzech największych pomiarów promienia). Każda ze zmiennych ma charakter ciągły.
	
\section{Charakterystyka i przetwarzenie danych}

	W celu przeprowadzenia procesu uczenia maszynowego, jednym z najistotniejszych kroków jakie należy podjąć jest wstępne zaznajomienie się z zestawem danych i jego analiza pod kątem rozkładu poszczególnych zmiennych oraz prawdopodobieństw. W tym celu wykorzystane zostało oprogramowanie JMP. 

	\subsection{Analiza rozkładu danych}
	
	\begin{figure}[!ht]
		\centering
		\includegraphics[width=0.8\linewidth]{Rysunki/Rozdzial2/diagnosis_distribution}
		\caption{Histogram rozkładu zmiennej odpowiedzi}
		\label{fig:diagnosisdistribution}
	\end{figure}
	
	\begin{figure}[!ht]
		\centering
		\includegraphics[width=0.7\linewidth]{Rysunki/Rozdzial2/variable_distribution}
		\caption{Przykłady histogramów zmiennych decyzyjnych}
		\label{fig:variabledistribution}
	\end{figure}
	
	
	Proces analizy rozkładu rozpoczęty został od przyjrzenia się zmiennej odpowiedzi (\textit{Diagnosis}). Rysunek 3.1 przedstawia uzyskany histogram, wraz z tabelą określającą ilość obserwacji danej klasy i współczynnik prawdopodobieństwa przynależności odpowiedzi do danej klasy. Zauważyć można, że dla użytego zestawu danych ilość zarejestrowano 357 obserwacji łagodnego raka piersi, a jego prawdopodobieństwo przynależności do klasy \textit{Benign} wynosi $\approx$ 62,7\%, natomiast do klasy \textit{Malignant} przynależało 212 obserwacji z prawodpodobieństwem $\approx$ 37,3\%.
		
	Podczas analizy histogramów zmiennych decyzyjnych, stwierdzono że znaczna ilość ma charakter prawostronnie skośny oraz występują dla nich obserwacje odstające, o czym informuje znajdujący się po prawej stronie histogramu wykres okienkowy (ang. \textit{box graph}), co przedstawiono na rysunku 3.2. Wyjątkiem okazała się zmienna \textit{Mean Largest Concave Points}), która mimo lekkiej skośności, okazała się nie posiadać obserwacji odstających. Na podstawie tych informacji stwierdzono, że aby przygotować dane w odpowiedni sposób do procesu uczenia należy przeprowadzić ich czyszczenie oraz normalizację rozkładu.
	
	\subsection{Czyszczenie i normalizacja rozkładu danych}
	
	Na pełny zestaw danych składa się 569 obserwacji. Podczas wstępnej analizy nie stwierdzono istnienia brakujących wartości, w związku z czym głównym problemem okazały się obserwacje odstające oraz skośności rozkładu. Do analizy obserwacji odstających wykorzystano wykresy okienkowe, gdzie oś Y reprezentowała zmienną odpowiedzi, natomiast oś X czyszczoną zmienną decyzyjną. Przykładowy wykres został przedstawiony na rysunku 3.3. Ze względu na bardzo małą ilość obserwacji zdecydowano się rozpocząć proces przystosowywania danych do uczenia poprzez normalizację ich rozkładu, aby zminimalizować lub wyeliminować konieczność usunięcia danych odstających. 
	
	\begin{figure}[!ht]
		\centering
		\includegraphics[width=0.7\linewidth]{Rysunki/Rozdzial2/box_graph}
		\caption{Przykład analizy obserwacji odstających dla poszczególnych klas zmiennej odpowiedzi}
		\label{fig:boxgraph}
	\end{figure}

	\newpage
	
	W pierwszym podejściu zdecydowano się na zastosowanie transformacji logarytmicznej dla wszystkich zmiennych decyzyjnych i porównanie charakterystyk uzyskanych rozkładów z oryginalnymi. Zmienna \textit{Mean largest concave points} okazała się posiadać rozkład bardzo zbliżony do standardowego, w związku z czym wyłączono ją z dalszej analizy normalizacji. Przykładowe wyniki przedstawiono na rysunku 3.4. Transformacja ta okazała się skutecznym rozwiązaniem jedynie dla następujących zmiennych:
	
	\begin{enumerate}
		\item \textit{Mean radius};
		\item \textit{Mean texture};
		\item \textit{Mean perimeter},
		\item \textit{Mean area};
		\item \textit{Mean smoothness};
		\item \textit{Mean symmetry};
		\item \textit{Std err texture};
		\item \textit{Std err smoothness};
		\item \textit{Std err compactness};
		\item \textit{Std err concave points};
		\item \textit{Mean largest texture};
		\item \textit{Mean largest smoothness};
		\item \textit{Mean largest compactness}.
	\end{enumerate}

	\begin{figure}[!ht]
		\centering
		\includegraphics[width=0.7\linewidth]{Rysunki/Rozdzial3/log}
		\caption{Porównanie rozkładu danych przed i po transformacji logarytmicznej.}
		\label{fig:log}
	\end{figure}
	
	\newpage

	W drugim kroku podjęto próbę wykorzystania transformacji pierwiastkiem sześciennym dla pozostałych zmiennych decyzyjych, ze względu na jej skuteczność dla danych o rozkładzie prawoskośnym. Rysunek 3.6. przedstawia porównanie rozkładu zmiennej \textit{Mean concavity} przed i po transfromacji pierwiastkiem sześciennym. Pomyślnie znormalizowano rozkład następujących zmiennych:
	
	\begin{enumerate}
		\item \textit{Mean compactness};
		\item \textit{Mean concavity};
		\item \textit{Mean concave points};
		\item \textit{Std err concavity};
		\item \textit{Mean largest radius};
		\item \textit{Mean largest perimeter};
		\item \textit{Mean largest concavity};
		\item \textit{Mean largest symmetry}.
	\end{enumerate} 

\begin{figure}[!ht]
	\centering
	\includegraphics[width=0.7\linewidth]{Rysunki/Rozdzial3/cube_root}
	\caption{Porównanie rozkładów danych przed i po zastosowaniu transformacji pierwiastkiem sześciennym.}
	\label{fig:cuberoot}
\end{figure}

	Ostatecznym krokiem okazało się zastosownie odwrotnej transformacji Arrheniusa, Niestety część z uzyskanych zmodyfikowanych zmiennych decyzyjnych zachowała częściowy skośny rozkład, jednak inne przetestowane transformacje, jak m.in. pierwiastek kwadratowy, potęga kwadratowa, logarytm x+1, logarytm dziesiętny, funkcja potęgowa, funkcja wykładnicza, przyniosły rezultaty porównywalne lub gorsze od uzyskanego w wyniku w/w odwrotnej transformacji Arrheniusa. Rysunek 3.6 przedstawia porównanie uzyskanych rozkładów.
	
	\newpage
	\begin{figure}[!ht]
		\centering
		\includegraphics[width=0.7\linewidth]{Rysunki/Rozdzial3/arrhenius}
		\caption{Porównanie uzyskanych rozkładów danych przed i po odwrotnej transformacji Arrheniusa.}
		\label{fig:arrhenius}
	\end{figure}
	
	Ze względu na bardzo małą ilość obserwacji, zdecydowano się na zachowanie wszystkich obserwacji odstających, aby zapobiec utracie informacji i zmianie uzyskanych w procesie normalizacji rozkładów.

\section{Szablony docelowych modeli dla zadanych danych eksperymentalnych}

Ze względu na dychotomiczny charakter zmiennej odpowiedzi, wybrany został przedstawiony poniżej zestaw metod dla których wykonano i przedstawiono testy praktyczne. Szablony struktury rozwiązań, takie jak np. wybór zmiennych uczestniczących w procesie uczenia, lub struktura sieci neuronowej zostały ustalone w sposób empiryczny z wykorzystaniem programu do uczenia maszynowego JMP. 

\subsection{Regresja logistyczna}

Badanie zależności w modelu regresji logistycznej odbyło się z wykorzystaniem wykresu wpływu zmiennej decyzyjnej na zmienną odpowiedzi opartego o p-wartość. Jako próg pozwalający na odrzucenie hipotezy zerowej (hipotezy o braku wypływu zmiennej na odpowiedź) przyjęto 0.05 jednostek. Rysunek 3.7 przedstawia w/w wykres wraz z p-wartościami dla poszczególnych zmiennych. Zauważyć można, że dla części zmiennych nie została wyznaczona p-wartość -- oznacza to, że część zmiennych jest ze sobą skorelowanych.

Pierwszym krokiem w wybraniu istotnych zmiennych było usunięcie zmiennych skorelowanych, drugim natomiast stopniowe usuwanie zmiennych o p-wartości powyżej określonego progu. Rysunek 3.8 przedstawia listę wraz z wykresem kolumnowym istotnych zmiennych, wymienionych poniżej:

\begin{enumerate}
	\item \textit{Log mean largest texture} (p-wartość 0.00000);
	\item \textit{Log mean largest compactness} (p-wartość 0.00000);
	\item \textit{Cube root mean largest symmetry} (p-wartość 0.00001);
	\item \textit{Arrhenius inverse std err symmetry} (p-wartość: 0.00005);
	\item \textit{Arrhenius inverse std radius} (p-wartość 0.00018);
	\item \textit{Cube root mean concave points} (p-wartość 0.00056);
	\item \textit{Cube root mean largest concavity} (p-wartość 0.00069);
	\item \textit{Log std err texture} (p-wartość 0.00252);
	\item \textit{Cube root mean largest perimeter} (p-wartość 0.00526);
	\item \textit{Log mean smoothness} (p-wartość 0.04867);
	\item \textit{Log mean radius} (p-wartość 0.04884).
\end{enumerate}

\begin{figure}[!ht]
	\centering
	\includegraphics[width=0.9\linewidth]{Rozdzial3/pvalue1}
	\caption{Wykres p-wartości dla całego zestawu zmiennych decyzyjnych.}
	\label{fig:pvalue1}
\end{figure}

\begin{figure}[!ht]
	\centering
	\includegraphics[width=0.9\linewidth]{Rozdzial3/pvalue2}
	\caption{Wykres i p-wartości istotnych zmiennych decyzyjnych}
	\label{fig:pvalue2}
\end{figure}

Dla wybranego zestawu zmiennych model osiągnął dokładność na poziomie $R^{2}$ = 0.9401. Zgodnie z macierzą pomyłek, 207 obserwacji typu \textit{Malignant} oraz 335 obserwacji \textit{Benign} zostało zaklasyfikowanych poprawnie. Oznacza to, że model uzyskał tylko 2 wyniki typu \textit{false-positive} (prawdopodobieństwo 0,6\%) i 5 wyników typu \textit{false-negative} (prawdopodobieństwo 2,4\%) dla danych treningowych. Ze względu na mały zestaw obserwacji, ryzyko przeuczenia jest znikome, w związku z czym nie wytypowano zestawu danych walidacyjnych. Rysunek 3.9 przedstawia krzywą charakterystyczną odbiornika dla modelu. 

\begin{figure}[!ht]
	\centering
	\includegraphics[width=0.6\linewidth]{Rozdzial3/roc}
	\caption{Krzywa charakterystyczna odbiornika (ROC) dla modelu regresji logistycznej}
	\label{fig:roc}
\end{figure}



\newpage
\subsection{Głęboka sieć neuronowa}

Do przygotowania sieci neuronowej wykorzystano zestaw zmiennych które zostały uznane za istotne w punkcje 3.3.1. Do uczenia zestaw danych został losowo podzielony na dane uczące i walidacyjne w propocji 80\% do 20\%. W wyniku prób i błędów, optymalny model uzyskano przy następującej strukturze:

\begin{itemize}
	\item [-] warstwa ukryta złożona z 5 neuronów o aktywacji tangensa hiperbolicznego;
	\item [-] warstwa ukryta złożona z 5 neuronów o aktywacji tangensa hiperbolicznego;
	\item [-] warstwa wyjściowa złożona z 2 neuronów.
\end{itemize}

Dla ziarna o wartości 1234 uzyskano model którego statystyka $R^{2}$ dla danych treningowych wyniosła 0.966268, natomiast dla danych testowych 0.9924547. Trafność dla losowo wybranego zestawu testowego wyniosła 100\%, natomiast dla danych uczących napotkano 5 przypadków \textit{false-negative} (prawdopodobieństwo 3\%) oraz 1 przypadek \textit{false-positive} (prawdopodobieństwo 0,4\%). Rysunki 3.10 oraz 3.11 przedstawiają krzywe charakterystyczne odbiornika dla zestawu testowego i walidacyjnego.

\begin{figure}[!ht]
	\begin{minipage}{0.48\textwidth}
			\centering
			\includegraphics[width=0.95\linewidth]{Rozdzial3/roc_test}
			\caption{Krzywa charakterysty-czna odbiornika dla zestawu testowego}
			\label{fig:roctest}
	\end{minipage}%
	\hspace{8pt}
	\begin{minipage}{0.48\textwidth}
			\centering
			\includegraphics[width=0.95\linewidth]{Rozdzial3/roc_valid}
			\caption{Krzywa charakterysty-czna odbiornika dla danych walidacyjnych}
			\label{fig:rocvalid}
	\end{minipage}
\end{figure}







\subsection{Maszyna wektorów nośnych}
\chapter{Shogun}

\section{Introductions}

Shogun is an open-source free machine learning library made in C++, accessible based on \textit{BSD 3-clause} license \cite{shogun:github}. It provides interfaces for various languages, such as Python, Ruby or C\#, but also allows users to use it in its native language. The library focuses on classification and regression problems.

\section{Data formats}

Base class allowing for loading data into Shogun is \textit{std::vector} present in the Standard Template Library (STL) of C++ language. As such, a user can use any mechanism of loading and parsing data, eg. from a file, network or another device, that at the end returns a vector containing the observations, as long as he provides it himself. One popular choice for storing training data is CSV file, for which Shogun provides support \cite{handsOnMachineLearning}. Still in that case data is a subject to several requirements:

\begin{itemize}
	\item \textbf{The file can only contain numeric data} - if any text data are present, preprocessing is needed to recode them into some numeric values (eg. one-hot encoding or regular subsequent numbers in terms of response variable classes). Sadly this means that data explanations cannot be stored alongside it.
	
	\item \textbf{Comma separator} - this becomes an issue if data is previously processed via other software, like Microsoft Excel or JMP, as it commonly uses semicolon as a separator. Despite the CSV format allows for several different separators, Shogun accepts only comma as a valid one.

	\item \textbf{Real values should use dot character as a decimal point} - this comes from the characteristics of C++ (and also other languages), and is required for the language to be able to automatically parse a value from text representation to numeric representation.
\end{itemize}

In order to read and parse data from CSV file, user needs to use \textit{shogun::CCSVFile} class, which result is then loaded into an object of \textit{shogun::SGMatrix}. Because of the fact that this class saved the stored data column-wise, to use them in the training process, user needs to transpose it, and then separate this matrix into two parts, one of which contains predictors, and the other labels. Sample code snippet of this procedure is shown on listing \ref{shogun:csv}. After correct data separation, the container needs to be transposed again so it will be accepted by the learning algorithm, and it needs to be wrapped into specifically designed classes called \textit{CDenseFeatures}, \textit{CMulticlassLabels} or \textit{CRegressionLabels}. This part was shown on listing \ref{shogun:csv2}.

\cppcode{Result/inc/shogun/csv.hpp}{Sample read and preprocessing of data from a CSV file}{shogun:csv}

\cppcode{Result/inc/shogun/shogunModels.hpp}{Repackaging data into desired containers}{shogun:csv2}

\section{Data processing and exploration methods}

\subsection{Normalizing}

The library provides a normalizing min-max mechanism that guarantees the data will be rescaled into a $\langle$ 0; 1 $\rangle$ range, with a class \textit{shogun::CRescaleFeatures}. An object of this class can be then used again for the new data corresponding to the previously learned predictors. Its two main methods are:

\begin{itemize}
	\item \textit{fit()} - teaches the normalizer object statistical characteristics of the provided data;
	\item \textit{transform()} - performs normalizing.
\end{itemize}

In case of some algorithms provided by Shogun, normalizing is one of the first steps executed automatically, so it is not always necessary to do this in preprocessing. Such information should be included in the documentation of a given method. Listing \ref{shogun:normalizer} shows how to use the aforementioned class. The class, aswell as the function shown on the listing perform the normalizing in-place, hence no need to override the object storing the data.

\cppcode{Result/inc/shogun/rescale.hpp}{Przykład funkcji wykonującej normalizację.}{shogun:normalizer}

\subsection{Dimensionality reduction}

Shogun gives to the user several algorithms of dimensionality reduction as following classes \cite{handsOnMachineLearning}:

\begin{itemize}
	\item \textbf{Principle Component Analysis} - class \textit{CPCA};
	\item \textbf{Kernel Principle Component Analysis} - class \textit{CKernelPCA};
	\item \textbf{Multidimensional Scaling} - class \textit{MultidimensionalScaling};
	\item \textbf{IsoMap} - class \textit{CIsoMap};
	\item \textbf{ICA} - class \textit{CFastICA};
	\item \textbf{Factor Analysis} - class \textit{CFactorAnalysis};
	\item \textbf{t-SNE} - class \textit{CTDistributedStochasticNeighborEmbedding}.
\end{itemize}

Each of the classes mentioned above operates by first learning the parameters of the training data by calling the \textit{fit()} function, and establishing the desired dimensions count (except for ICA). Trained reductor object can be used to reduce the dimensionality by \textit{apply\char`_to\char`_feature\char`_vector()} method, that returns transformed data, or in case of ICA, Factor Analysis and t-SNE, by \textit{transform()} method, which result needs to be casted into a pointer to \textit{CDenseFeatures}. Unfortunately, using any of the reductor objects requires creating a new copy of the data object during transformation, instead of performing it in-place. Listing \ref{shogun:reduction} shows an example based on class \textit{CKernelPCA}.

\cppcode{Result/inc/shogun/pca_reduction.hpp}{Dimensionality reduction using Kernel PCA \cite{handsOnMachineLearning}.}{shogun:reduction}

\subsection{L1 and L2 Regularisation}

In Shogun, regularisation is an integral part of a machine learning model, and as such, it is always performed by given model. Each model defines itself which kind of regularization it performs, and this cannot be changed.

\section{Machine Learning Models}
\subsection{Linear Regression}
One of the fundamental machine learning algorithms provided by the Shogun library is linear regression, performed by the \textit{CLinearRidgeRegression} class. As the name suggests, it utilizes Ridge Regression, configured at the model creation stage. Listing \ref{shogun:linear} shows how to adjust the model.

\newpage
\cppcode{Result/inc/shogun/linear.hpp}{Linear regression example in Shogun}{shogun:linear}

\subsection{Logistic Regression}
Shogun library contains implementation of multiclass logistic regression via \textit{CMulticlassLogisticRegression} class. It also provides configurable regularisation. Listing \ref{shogun:logistic} shows how to use it.

\cppcode{Result/inc/shogun/logistic.hpp}{Logistic regression example in Shogun}{shogun:logistic}

\subsection{Support Vector Machine}
Similarly to logistic regression, Shogun provides the multiclass Support Vector Machine model implementation for classification tasks, as \textit{CMulticlassLibSVM} class. It contains several configurable parameters, along with the choice of kernel itself. Listing \ref{shogun:svm} shows an example of use.


\hspace{20px}
\cppcode{Result/inc/shogun/svm.hpp}{Support Vector Machine example in Shogun}{shogun:svm}

\subsection{K-Nearest Neighbours}

K-Nearest Neighbours algorith is available as \textit{CKNN} class. It allows user to select the method of calculating distances by passing a correct object, and the count of neighbours considered the nearest. The main available distance metrics are: Euklidean, Hamming, Manhattan and cosinus similarity. In comparison to other methods, it does not require the configuration of hyperparameters, allowing the use of it in crossvalidation without problems. Listing \ref{shogun:knn} presents an example configuration and use of KNN algorithm using Euclidean distance.

\cppcode{Result/inc/shogun/knn.hpp}{KNN algorithm example in Shogun}{shogun:knn}

\subsection{Ensemble algorithms}
\subsubsection{Gradient boosting}

Implementation of ensemble algorithm using gradient boosting is adjusted only for regression models. The class \textit{CStochasticGBMachine} is responsible for its execution. It allows the user to configure several parameters, some of which are:

\begin{itemize}
	\item base algorithm;
	\item loss function;
	\item iteration count;
	\item learning coefficient;
	\item fraction of vectors to choose at each iteration.
\end{itemize}

Listing \ref{shogun:gb} shows a method of creating such model using binary decisive tree for regression and classification (implemented by \textit{CCARTree} class) as a base algorithm.

\cppcode{Result/inc/shogun/gb.hpp}{Gradient boosting example in Shogun}{shogun:gb}

\subsubsection{Random forest}
The random forest method is available in the Shogun library via the \textit{CRandomForest} class. On the contrary to gradient boosting, its implementation allows also for classification tasks. Some of the main configurable parameters are:

\begin{itemize}
	\item number of trees;
	\item number of batches to which the data should be separated;
	\item algorithm of the result selection;
	\item type of the problem;
	\item continuity of the regressors.
\end{itemize}

Listing \ref{shogun:rf} shows how to create and configure random forest model for cosinus approximation.

\cppcode{Result/inc/shogun/rforest.hpp}{Przykład użycia metody losowego lasu}{shogun:rf}

\subsection{Neural network}

First step in creating a neural network with this library is to configure the network topology, using \textit{CNeuralLayers} object. It consists of a range of methods, which create layers with given activation function:

\begin{itemize}
	\item \textit{input()} - input layer with a certain number of inputs;
	\item \textit{logistic()} - fully connected sigmoid function layer;
	\item \textit{linear()} - fully connected linear function layer;
	\item \textit{rectified\char`_linear()} - fully connected ReLU layer;
	\item \textit{leaky\char`_rectified\char`_linear} - fully connected Leaky ReLU layer;
	\item \textit{softmax} - fully connected softmax layer;
\end{itemize}

The order in which we call the methods is important, because it decides about the order of layers within the model. After completing the configuration process, it is possible to confirm the architecture by creating another object via the \textit{done()} method, and using it to initialize \textit{CNeuralNetwork} class. In order to connect the layers, the \textit{quick\char`_connect} method needs to be called, and subsequently, the weights have to be initialized via \textit{initialize\char`_neural\char`_network()} function. It accepts a parameter describing a Gaussian distribution used to initialize weights.

The next step is to configure the optimizer using \textit{set\char`_optimization()} method. The class \textit{CNeuralNetwork} supports optimizing using the steepest descent method and Broyden-Fletcher-Goldfarb-Shannon method. This model has in-built L2 regularization, which can be further configured, similarly to other parameters such as learning coefficient, epoch count, convergence criterion for the loss function, or the size of batches. Unfortunately, user cannot select the loss function in question, since it is established automatically based on the label type. Listing \ref{shogun:nn} shows the full process of building, configuring and teaching a network. Unfortunately, due to the lack of hyperbolic tangential function implementation, a reLU function was used in its stead, which affects the achieved results.

\cppcode{Result/inc/shogun/neural.hpp}{Example of neural network use in Shogun.}{shogun:nn}

\section{Model analysis methods}

\subsection{Mean squared error}

Calculation of mean squared error in the Shogun library comes down to creating an object of type \textit{CMeanSquaredError} as an argument to the \textit{some<>()} template function. It is returned as a pointer to the object. In order to acquire the value of the error for given data, user needs to call the \textit{evaluate()} method, passing prediction results and observed output. Listing \ref{shogun:mse} presents the detailed way to use it.


\cppcode{Rozdzial4/shogun-mse.cpp}{Example of calculating mean squared error in Shogun\cite{handsOnMachineLearning}}{shogun:mse}

\subsection{Mean absolute error}

Mean absolute error calculation is carried out by the \textit{CMeanAbsoluteError} class, which serves as an evaluator. It is created by a call to \textit{some<>()} function and used for evaluation by calling the \textit{evaluate()} method, providing the results of a model and expected labels of regression or classification. Listing \ref{shogun:mae} shows how to use the aforementioned class.

\cppcode{Rozdzial4/shogun-mae.cpp}{Example of mean absolute error calculation using Shogun\cite{handsOnMachineLearning}.}{shogun:mae}

\subsection{Logarithmic loss function}

Logarithmic loss function is possible to calculate using the Shogun library using \textit{CLogLoss} class, however the public interface of said class indicate that it should be rather used by a model than directly by the user. It provides \textit{get\char`_square\char`_grad()} method which allows for calculation of the squared gradient between prediction and expected result. The use case is presented on listing \ref{shogun:log}.

\cppcode{Rozdzial4/shogun-log.cpp}{Example of use for \textit{CLogLoss} class.}{shogun:log}

\subsection{$R^2$ metric}

The Shogun library does not provide implementation for the $R^2$ metric as such, so despite the ability to use the in-built method of mean squared error calculation, the variance of output needed for calculating the metric needs to be acquired from a user-provided mechanism. Listing \ref{shogun:verify} shows a function that verifies the correctness of models described in previous paragraphs by calculating the $R^2$ metric.

\cppcode{Result/inc/shogun/verify.hpp}{Example of $R^2$ metric calculation.}{shogun:verify}

\subsection{Accuracy}

In order to calculate the accuracy metric for regression tasks, this library provides \textit{CMulticlassAccuracy} class. It also allows for getting a misclassification matrix. The \textit{CAccuracyMeasure} class mentioned in \cite{handsOnMachineLearning} however was not found, which indicates it was removed from the library. Listing \ref{shogun:acc} shows how to use \textit{CMulticlassAccuracy}.

\cppcode{Rozdzial4/shogun-acc.cpp}{Example of accuracy calculation in Shogun.}{shogun:acc} 

\subsection{Precision, Recall and F1 metric}

The \cite{handsOnMachineLearning} handbook mentions \textit{CRecallMeasure} and \textit{CF1Measure} classes that are supposed to allow the user to calculate the recall and F1 metric of the model, however while working with the library, their definitions were not found, neither any other classes of similar functionality, and as such it was assumed that the Shogun library does not provide the implementation for them. The motivation to touch on this topic was to inform the reader of possible discrepancies between available sources and current state of the library. 

\subsection{ROC curve}

The Shogun library provides implementation to calculate ROC curve as the \textit{CROCEvaluation} class. Listing \ref{shogun:roc} shows the way how to use it.

\cppcode{Rozdzial4/shogun-roc.cpp}{ROC curve calculation example for Shogun.}{shogun:roc}

\subsection{K-fold cross-validation}

Cross-validation inside the Shogun library is a complex mechanism, which in order to use it requires a decision tree of available parameters, represented by the \textit{CModelSelectionParameters} class. The user can select the model and evaluation criterion for it by creating objects of selected classes and providing them to the \textit{CCrossValidation} instance constructor. The next step is to create an instance of \textit{CGridSearchModelSelection} which will select parameters. The last phase is to configure the desired model and complete the teaching process. Precise example of the whole mechanism was presented on listing \ref{shogun:cross}.

\cppcode{Result/inc/shogun/cross.hpp}{Preparation of multiclass classification via linear regression with cross-validation in Shogun.}{shogun:cross}

\section{Documentation and sources availability}

Internet sources such as community forums focus on using Shogun with different programming languages such as eg. Python, however using its source code available on GitHub \cite{shogun:github} it is possible to generate examples of its use in C++ language aswell in the \textit{examples} directory. Those examples need to be build using a specific Python script contained within the repository, that generates listings of codes in desired language in JSON files. Unfortunately, they present the use of the library in very unnatural, procedurally generated manner, causing them to be useless when attempting to use the actual project's API. In addition, the only form of project documentation is restricted to the comments within the source code, making the user wander through the repository in search of the needed information.

The Shogun is one of the libraries described within the \textit{"Hands-On Machine Learning with C++"}\cite{handsOnMachineLearning} handbook, which both introduces the reader to basic functionalities of the library, aswell as summarizes the basics of machine learning theory in order to use it. Majority of various model types examples within this book is equipped with listings of main code snippets for the Shogun library. It is important to mention however, that they are different from the examples generated by the build script, showing the use of actual API of the library. During the making of this chapter, this handbook proved to be the only valuable source on the topic.
\chapter{Biblioteka Shark-ML}

\section{Wprowadzenie}

Shark-ML to biblioteka uczenia maszynowego dedykowana dla języka C++. Posiada ono otwarte źródło, i udostępniana jest na podstawie licencji \textit{GNU Lesser General Public License}. Głównymi aspektami na których skupia się ta biblioteka są problemy liniowej i nieliniowej optymalizacji (w związku z czym posiada ona część funkcjonalności biblioteki do algebry liniowej), maszyny jądra (np. maszyna wektorów nośnych) i sieci neuronowe. \cite{shark} Podmiotami udostępniającymi bibliotekę jest Uniwersytet Kopenhagi w Danii, oraz Instytut Neuroinformatyki z Ruhr-Universitat Bochum w Niemczech.

\section{Formaty źródeł danych}

Biblioteka operuje na własnych reprezentacjach macierzy i wektorów, które tworzone są poprzez opakowywanie surowych tablic. Mechanizm ten jest identyczny jak w przypadku pozostałych z omawianych bibliotek, co daje użytkownikowi dużą dowolność co do sposobu przechowywania danych i mechanizmu ich odczytywania. Posiada ona także dedykowany parser dla plików w formacie CSV, lecz zakłada on obecność w pliku jedynie danych numerycznych. Do jego użycia należy użyć klasy kontenera \textit{ClassificationDataset} oraz metody \textit{importCSV} która zapisuje odczytane dane do wcześniej wspomnianego obiektu poprzez mechanizm zwracania przez parametr. Jeden z argumentów funkcji określa która z kolumn zawiera zmienną decyzyjną, dzięki czemu biblioteka jest w stanie od razu oddzielić dane wejściowe od kolumny oczekiwanych wartości. Artykuł ,,Classification with Shark-ML machine learning library''\cite{shark:http} dostępny na platformie GitHub pokazuje także, jak pobrać dane w postaci formatu CSV z internetu z pomocą API biblioteki \textit{curl}, i przetworzyć je do formy akceptowanej przez Shark-ML, co zostało przedstawione na listingu \ref{shark:http}. W aktualnej wersji biblioteki znalazły się także wbudowane funkcje pobierania danych współpracujące z protokołem HTTP.

\newpage
\cppcode{Rozdzial5/shark-csv-http.cpp}{Pobranie danych do uczenia z serwera HTTP \cite{shark:http}}{shark:http}

W celu opakowania danych zawartych w kontenerach biblioteki standardowej języka C++ do obiektów akceptowanych przez bibliotekę Shark-ML, konieczne jest wykorzystanie specjalnych funkcji adaptorowych, do których przekazywany jest wskaźnik na dane w postaci surowej tablicy, wraz z oczekiwanymi wymiarami wynikowej macierzy / wektora. Sposób opakowania danych pokazano na listingu \ref{shark:adaptor}

\cppcode{Rozdzial5/shark-adaptor.cpp}{Sposób opakowywania danych do przetwarzania przez Shark-ML \cite{handsOnMachineLearning}}{shark:adaptor}

\section{Metody przetwarzania i eksploracji danych}

Biblioteka Shark-ML implementuje normalizacje jako klasy treningowe dla modelu \textit{Normalizer}, udostępniając użytkownikowi trzy możliwe do wykorzystania klasy:

\begin{itemize}
	\item \textit{NormalizeComponentsUnitInterval} - przetwarza dane tak aby mieściły się w przedziale jednostkowym;
	\item \textit{NormalizeComponentsUnitVariance} - przelicza dane aby uzyskać jednostkową wariancję, i niekiedy także średnią wynoszącą 0.
	\item \textit{NormalizeComponentsWhitening} - dane przetwarzane są w sposób zapewniający średnią wartość wynoszącą zero oraz określoną przez użytkownika wariancję (domyślnie wariancja jednostkowa).
\end{itemize}

Opierają się one o użycie metody \textit{train()} na obiekcie normalizera, aby odpowiednio go skonfigurować do przetwarzania zarówno danych testowych, jak i wszystkich innych danych które użytkownik ma zamiar wprowadzić do modelu. Istnieją ponadto klasy trenerów które w przeciwieństwie do powyższych, nie używają metody \textit{train()}, czego przykładem jest klasa PCA, wykorzystywana do redukcji wymiarowości zestawu danych. Listingi \ref{shark:preprocessing} oraz \ref{shark:pca} ukazują sposób wykorzystania powyżej opisanych mechanizmów.

\cppcode{Rozdzial5/shark-preprocessing.cpp}{Wstępne przetwarzanie danych do uczenia \cite{shark:http}}{shark:preprocessing}
\cppcode{Rozdzial5/shark-dimension-reduction.cpp}{Redukcja wymiarowości danych z wykorzystaniem klasy PCA i enkodera}{shark:pca}

Dodatkowymi funkcjami jest możliwość przemieszania danych, i wydzielenia fragmentu jako dane testowe za pomocą metody \textit{shuffle()} klasy \textit{ClassificationDataset} oraz funkcji \textit{splitAtElement()}.

\section{Modele uczenia maszynowego}

\subsection{Regresja liniowa}

Jednym z podstawowych modeli oferowanych przez niniejszą bibliotekę jest regresja liniowa. Do celów jej reprezentacji dostępna jest klasa \textit{LinearModel}, oferująca rozwiązanie problemu w sposób analityczny za pomocą klasy trenera \textit{LinearRegression}, lub podejście iteracyjne implementowane przez klasę trenera \textit{LinearSAGTrainer}, wykorzystujące iteracyjną metodę gradientu średniej statystycznej (ang. \textit{Statistic Averagte Gradient, SAG}). W przypadku bardziej skompilowanych regresji, gdzie może nie istnieć rozwiązanie analityczne, istnieje możliwość zastosowania podejścia iteracyjnego z użyciem optymalizatora wybranego przez użytkownika. Metoda ta sprowadza się to uczenia optymalizatora z wykorzystaniem funkcji straty, a następnie załadowanie uzyskanych wag do modelu regresji. Parametry modelu możliwe są do odczytania z wykorzystaniem metod \textit{offset()} i \textit{matrix()} lub metody \textit{parameterVector()}. Na listingu \ref{shark:linear} ukazane zostało wykorzystanie podejścia iteracyjnego, natomiast listing \ref{shark:linear2} przedstawia metodę analityczną.

\cppcode{Rozdzial5/shark-linear.cpp}{Przykład regresji liniowej z wykorzystaniem optymalizatora spadku gradientowego \cite{shark:linear}}{shark:linear}

\cppcode{Rozdzial5/shark-linear2.cpp}{Przykład regresji liniowej z wykorzystaniem trenera analitycznego \cite{handsOnMachineLearning}}{shark:linear2}

\subsection{Liniowa analiza dyskryminacyjna}

Model liniowej analizy dyskryminacyjnej (ang. \textit{Linear Discriminant Analysis, LDA}) w przypadku biblioteki Shark-ML opiera się o rozwiązanie analityczne, poprzez konfigurację klasy modelu \textit{LinearClassifier} przez klasę treningową \textit{LDA}, wykorzystując funkcję \textit{train()}. Predykcje uzyskiwane są za pomocą wywołania obiektu modelu jak funkcji (użycie operatora ()) przekazując mu dane uzyskane z ClassificationDataset za pomocą metody \textit{inputs()}. Szczegóły implementacyjne zamieszczone zostały na listingu \ref{shark:lda}.

\newpage
\cppcode{Rozdzial5/shark-lda.cpp}{Przykład klasyfikacji z wykorzystaniem modelu LDA \cite{shark:lda}}{shark:lda}

\subsection{Regresja logistyczna}

Mechanizm regresji logistycznej dostępny w bibliotece Shark-ML z natury rozwiązuje problem regresji binarnej. Istnieje jednak możliwość przygotowania wielu klasyfikatorów, w ilości wyrażonej wzorem:

\begin{equation}
	\frac{N(N-1)}{2}	
	\label{multiclass}
\end{equation}

gdzie N oznacza ilość klas występujących w problemie. Utworzone klasyfikatory następnie są złączane w jeden za pomocą odpowiedniej konfiguracji obiektu \textit{OneVersusOneClassifier}, rozwiązując problem klasyfikacji wieloklasowej. W tym celu zestaw danych należy iteracyjnie podzielić na podproblemy o charakterystyce binarnej za pomocą wbudowanej funkcji \textit{binarySubProblem()} przyjmującej zestaw danych i klasy. Nauczanie poszczególnych modeli realizowane jest poprzez klasę trenera \textit{LogisticRegression}. Po zakończeniu trenowania okreslonej partii pomniejszych modeli, są one ładowane do głównego modelu. Wykorzystanie gotowego klasyfikatora wieloklasowego nie różni się od sposobu użycia modelu uzyskanego np. w klasyfikacji liniowej. Listing \ref{shark:logistic} prezentuje funkcję budującą model logistycznej regresji wieloklasowej, natomiast listing \ref{shark:logistic2} prezentuje sposób utworzenia prostego modelu dla problemu binarnego.

\cppcode{Rozdzial5/shark-logistic.cpp}{Przykład funkcji tworzącej model wieloklasowej regresji logistycznej \cite{handsOnMachineLearning}}{shark:logistic}

\cppcode{Rozdzial5/shark-logistic2.cpp}{Przykład prostej binarnej regresji logistycznej}{shark:logistic2}

\subsection{Maszyna wektorów nośnych}

Jednym z bardzo istotnych z perspektywy zastosowania biblioteki Shark-ML oferowanych przez nią metod uczenia maszynowego jest maszyna wektorów nośnych stanowiąca rodzaj tzw. modeli jądra (ang. \textit{kernel model}). Opiera się ona na wykonaniu regresji liniowej w przestrzeni cech określonych przez wykorzystany kernel. Podobnie jak w przypadku regresji logistycznej, API biblioteki umożliwia wykonanie klasyfikacji dla przypadku binarnego, natomiast rozwiązanie przy jej użyciu problemu wieloklasowego wymaga kombinacji instancji maszyn wektorów nośnych w model złożony, czego można dokonać przy pomocy klasy \textit{OneVersusOneClassifier} oraz ilości klas wyrażonej wzorem \ref{multiclass}. Zgodnie z charakterystyczną cechą tej biblioteki, użycie metody podzielone jest na utworzenie instancji modelu oraz obiektu klasy trenera, która go konfiguruje w procesie uczenia. W tym celu dostępne są dla użytkownika klasy:

\begin{itemize}
	\item \textit{GaussianRbfKernel} - odpowiada za obliczenie podobieństwa między zadanymi cechami wykorzystując funkcję bazową \textit{ang. Radial Basis Function, RBF};
	\item \textit{KernelClassifier} - funkcja realizująca regresję liniową wewnątrz przestrzeni określonej przez jądro;
	\item \textit{CSvmTrainer} - klasa trenera realizująca uczenie w oparciu o skonfigurowane parametry;
\end{itemize}

Do parametrów pozwalających na konfigurację modelu należą m.in.:

\begin{itemize}
	\item przepustowość modelu - podawana w konstruktorze \textit{GaussianRbfKernel} jako liczba z przedziału $\langle 0 ; 1 \rangle$;
	\item regularyzacja - podawana jako liczba rzeczywista w konstruktorze \textit{CSvmTrainer}, domyślnie maszyna wektorów nośnych używa kary typu \textit{1-norm penalty} za przekroczenie docelowej granicy;
	\item bias - flaga binarna (bool) określająca czy model ma używać biasu, podawana w konstruktorze \textit{CSvmTrainer};
	\item \textit{sparsify} - parametr określający czy model ma zachować wektory które nie są nośne, dostępny przez metodę \textit{sparsify()} trenera;
	\item minimalna dokładność zakończenia nauczania - pozwala wyspecyfikować precyzję modelu, jest dostępna jako pole struktury zwracane przez metodę \textit{stoppingCondition()} klasy trenera;
	\item wielkość cache - ustawiana za pomocą funkcji \textit{setCacheSize()} trenera;
\end{itemize}

Sposób użycia modelu jest identyczny jak w przypadku pozostałych modeli, poprzez operator wywołania funkcji - (). Listing \ref{shark:svm} ukazuje przykład utworzenia i skonfigurowania modelu na podstawie wpisów dostępnych w dokumentacji biblioteki, natomiast listing \ref{shark:svm} przedstawia sposób utworzenia maszyny wektorów nośnych dla problemów wieloklasowych wewnątrz funkcji przyjmującej zestawy danych uczących i testowych.

\cppcode{Rozdzial5/shark-svm.cpp}{Przykład maszyny wektorów nośnych dla problemu binarnego \cite{shark:svm}}{shark:svm}

\cppcode{Rozdzial5/shark-svm2.cpp}{Przykład maszyny wektorów nośnych dla problemu wieloklasowego \cite{handsOnMachineLearning}}{shark:svm2} 

\subsection{Sieć neuronowa}
\subsection{Neuronowa sieć splotowa}

Biblioteka Shark-ML oprócz omówionych wyżej modeli, wspiera także modele do przetwarzania grafiki w postaci neuronowych sieci splotowych. Leżą jednak one poza zakresem niniejszej pracy, w związku z czym nie zostaną one dokładnie omówione.

\section{Metody analizy modeli}

https://www.shark-ml.org/doxygen\texttt{\char`_}pages/html/group\texttt{\char`_}\texttt{\char`_}lossfunctions.html

SquaredLoss
ZeroOneLoss

\section{Dostępność dokumentacji i źródeł wiedzy}

Biblioteka Shark-ML posiada skróconą dokumentację dostępną na głównej stronie internetowej projektu, wraz z przykładowymi plikami źródłowymi dołączonymi do repozytorium. Jest ona także wspomniana w książce ,,Hands-On Machine Learning with C++'', przedstawiającej sposoby użycia wybranych funkcjonalności. Kwestią wyróżniającą ją natomiast na tle pozostałych bibliotek omówionych w ramach niniejszej pracy jest fakt, że jest ona dedykowana dla języka C++, w związku z czym dużo łatwiej dostępne są wątki społecznościowe i artykuły omawiające realizację różnorodnych typów modeli z jej użyciem, oraz oferując przykładowy kod źródłowy.

\section{Przykłady testowe}
\subsection{Regresja liniowa}
\subsection{Maszyna wektorów nośnych}
\subsection{Sieć neuronowa}


%========================================================================================
% Dodatki
%========================================================================================
\begin{appendix}
\appendix
	%\input{Dodatki/dodatek_a.tex}
	%\input{Dodatki/dodatek_b.tex}
\end{appendix}

%========================================================================================
% Literatura
%========================================================================================

\begin{flushleft}
	\printbibliography[title={Bibliografia}]
\end{flushleft}

\end{document}
%========================================================================================
% Koniec dokumentu
%========================================================================================
