
% ------------------------------------------------------------------------
%   Kropki po numerach sekcji, podsekcji, itd.
%   Np. 1.2. Tytuł podrozdziału
% ------------------------------------------------------------------------
\makeatletter
    \def\numberline#1{\hb@xt@\@tempdima{#1.\hfil}}                      %kropki w spisie treści
    \renewcommand*\@seccntformat[1]{\csname the#1\endcsname.\enspace}   %kropki w treści dokumentu
\makeatother

\makeatother
% ------------------------------------------------------------------------
% Definicje
% ------------------------------------------------------------------------
\def\nonumsection#1{%
    \section*{#1}%
    \addcontentsline{toc}{section}{#1}%
    }
\def\nonumsubsection#1{%
    \subsection*{#1}%
    \addcontentsline{toc}{subsection}{#1}%
    }
\reversemarginpar %umieszcza notki po lewej stronie, czyli tam gdzie jest więcej miejsca
\def\notka#1{%
    \marginpar{\footnotesize{#1}}%
    }
%\def\mathcal#1{%
%    \mathscr{#1}%
%    }

\newcommand{\myemptypage}{ \newpage  \thispagestyle{empty}~\newpage}
\usepackage{makecell}

%-------------------------------------------------------------------------
% listingi
%-------------------------------------------------------------------------
\definecolor{commentColor}{rgb}{0,0.643,0}
\definecolor{keywordColor}{rgb}{0,0,0.835}
\lstdefinestyle{praca}{basicstyle=\footnotesize\ttfamily,
                        keywordstyle=\color{keywordColor},
                        commentstyle=\color{commentColor},
                        numbers=left,
                        stepnumber=1,
                        numberstyle=\scriptsize,
                        numbersep=10pt,
                        basewidth=0.5em,
                        extendedchars=true,
                        frame=tb}

\newcommand{\cppcode}[3]{\vspace{8pt}\lstinputlisting[caption=#2, style=praca, language=C++, label=#3, xleftmargin=.02\textwidth, xrightmargin=.02\textwidth]{#1}}

%-------------------------------------------------------------------------
% stopka i nagłówek
%-------------------------------------------------------------------------
\setlength{\headheight}{15pt}

\pagestyle{fancy}
\renewcommand{\chaptermark}[1]{\markboth{#1}{}}
\renewcommand{\sectionmark}[1]{\markright{#1}{}}

\fancyhf{}
\fancyhead[LE,RO]{\thepage}
\fancyhead[RE]{\textit{\nouppercase{\leftmark}}}
\fancyhead[LO]{\textit{\nouppercase{\rightmark}}}

\fancypagestyle{plain}{ %
\fancyhf{}
\renewcommand{\headrulewidth}{0pt}
\renewcommand{\footrulewidth}{0pt}}

% ------------------------------------------------------------------------
% Inne
% ------------------------------------------------------------------------
\frenchspacing
\setlength{\parskip}{3pt}           	%odstęp pomiędzy akapitami
%\linespread{1.49}                    	%odstęp pomiędzy liniami (interlinia)
\setcounter{tocdepth}{3}
\setcounter{secnumdepth}{3}


% ------------------------------------------------------------------------
% Polskie podpisy
% ------------------------------------------------------------------------
\renewcommand{\figurename}{Figure}
\renewcommand{\tablename}{Table}

% ------------------------------------------------------------------------
% Bibliografia
% ------------------------------------------------------------------------
%\bibliographystyle{unsrt}					% kolejność według użycia
%\bibliographystyle{plain}					% kolejność alfabetyczna
\bibliography{Bibliografia/bibliografia.bib}
  
  

%==========================================================================================
% Deklaracja fontow kapitalikowych z kodowaniem T1
%==========================================================================================
\DeclareFontShape{T1}{lmr}{bx}{sc} { <-> ssub * cmr/bx/sc }{}
\DeclareFontShape{T1}{lmr}{bx}{scit}{<-> ssub * cmr/bx/scsl}{}
%==========================================================================================
% Inne deklaracje
%==========================================================================================
