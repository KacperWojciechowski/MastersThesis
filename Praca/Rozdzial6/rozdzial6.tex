\chapter{Biblioteka Dlib}

\section{Wprowadzenie}

Jest to biblioteka do uczenia maszynowego napisana w nowoczesnym C++, o zastosowaniu przemysłowym oraz naukowym \cite{Dlib09}. Podobnie jak poprzednio omawiane biblioteki, posiada ona otwarte źródło na licencji Boost Software Licence \cite{dlib:license}. Do dziedzin wykorzystujących wyżej wspomnianą bibliotekę należą robotyka, systemy wbudowane, telefonia komórkowa oraz śrowodiska o dużej wydajności obliczeniowej. Kod źródłowy biblioteki opatrzony jest testami jednostkowymi, co pozwala na łatwiejsze utrzymanie jakości dostarczanego rozwiązania. Ciekawym aspektem jest fakt, że Dlib stanowi nie tylko bibliotekę, lecz zestaw narzędzi, oferujący funkcjonalności wykraczające także poza dziedzinę uczenia maszynowego.

\section{Formaty źródeł danych}

Do reprezentacji wektora w bibliotece Dlib wykorzystywane są kontenery z biblioteki szablonów STL języka C++. Dodatkowo, istnieje możliwość ich inicjalizacji za pomocą operatora przecinka, oraz opakowania surowej tablicy (ang. \textit{raw array}). Oznacza to, że podobnie jak w przypadku biblioteki Shogun, dane mogą być przekazywane do programu wykorzystującego Dlib w dowolny sposób zapewniający umieszczenie ich np. w surowej tablicy do późniejszego przetworzenia na obiekty akceptowane przez bibliotekę. Metoda ta działa także z kontenerami biblioteki STL, które pozwalają na dostęp do surowych danych przy użyciu metody \textit{data()}. Tak samo jak poprzednio, występuje tu wsparcie dla formatu CSV obwarowanego tymi samymi ograniczeniami co dla Shogun. Za wspomniane wsparcie odpowiada przeładowany operator strumienia współpracujący z klasą \textit{std::ifstream} biblioteki standardowej C++. Przykładowy kod wykorzystujący opisany mechanizm zamieszczony został na listingu \ref{dlib:csv}.

\cppcode{Rozdzial6/csv.cpp}{Fragment kodu ilustrujący sposób odczytu z pliku w formacie CSV \cite{handsOnMachineLearning}}{dlib:csv}

\section{Metody przetwarzania i eksploracji danych}

\subsection{Normalizacja}
Biblioteka udostępnia normalizację danych poprzez standaryzację, realizowaną przez klasę \textit{Dlib::vector\texttt{\char`_}normalizer}. Głównym warunkiem ograniczającym zastosowanie jej jest fakt, że nie można w niej umieścić całego zestawu danych treningowych na raz, co wymusza podział obserwacji na osobne wektory, a następnie umieszczenie ich w kontenerze \textit{std::vector} do dalszego przetwarzania. Przykład funkcji normalizującej przedstawiono na listingu \ref{dlib:normalizer}.

\cppcode{Result/inc/dlib/normalizer.hpp}{Funkcja normalizująca}{dlib:normalizer}

\subsection{Redukcja wymiarowości}

\subsubsection{PCA}

Implementacja metody PCA w bibliotece Dlib oferowana jest za pośrednictwem klasy \textit{dlib::vector\char`_normalizer\char`_pca}, która oprócz samej redukcji wymiarowości wykonuje także wcześniej automatycznie proces normalizacji danych. Bywa to przydatne, gwarantując że redukcja będzie przeprowadzana zawsze na odpowiednio przygotowanych wartościach obserwacji. Listing \ref{dlib:pca} przedstawia funkcję używającą wyżej wymienioną metodę.

\cppcode{Result/inc/dlib/pca.hpp}{Przykład redukcji wymiarowości z użyciem metody PCA}{dlib:pca}

\subsubsection{Liniowa analiza dyskryminacyjna}

Drugim z oferowanych algorytmów redukcji wymiarowości zawartych w Dlib jest algorytm liniowej analizy dyskryminacyjnej. Jest on dostępny pod postacią funkcji \textit{dlib::compute\char`_lda\char`_transform}, która przekształca macierz zawierającą dane wejściowe do macierzy transformacji danych. Ze względu na nadzorowany charakter algorytmu, konieczne jest także przekazanie wartości zmiennych odpowiedzi, natomiast same dane, w przeciwieństwie do metody PCA, mogą być zawarte w pojedynczym obiekcie macierzy. Redukcja odbywa się poprzez wymnożenie otrzymanej macierzy przez transponowany wiersz zawierający próbkę. Szczegółowe zastosowanie algorytmu przedstawiono na listingu \ref{dlib:lda}

\cppcode{Result/inc/dlib/lda.hpp}{Przykład redukcji wymiarowości z użyciem algorytmu LDA}{dlib:lda}

\subsubsection{Mapowanie Sammona}

Całość algorytmu mapowania Sammmona implementowana jest za pomocą klasy \textit{dlib::sammon\char`_projection}, i ogranicza się do utworzenia jej instancji. Wykorzystując metodę należy przekazać do utworzonego obiektu za pomocą operatora wywołania funkcji wektor danych, oraz oczekiwaną ilość wymiarów, otrzymując przekształcone dane. W związku z powyższym, funkcja realizująca redukcję z użyciem wyżej wymienionej metody sprowadza się do wykorzystania dwóch linii. Dokładny sposób jej użycia pokazano na listingu \ref{dlib:sammmon}

\cppcode{Result/inc/dlib/sammon.hpp}{Przykład redykcji wymiarowości z użyciem mapowania Sammmona}{dlib:sammon}

\section{Modele uczenia maszynowego}

\subsection{Regresja liniowa}
Biblioteka Dlib posiada pośrednią realizację modelu regresji liniowej. Wykorzystuje ona technikę brzegowej regresji jądra, przekazując w formie kernela jądro liniowe. Następnie przeprowadzany jest trening, zapisując docelowy model w postaci funkcji decyzyjnej. Listing \ref{dlib:linear} przedstawia szczegóły powyższego mechanizmu.

\cppcode{Result/inc/dlib/linear.hpp}{Przykład regresji liniowej w Dlib}{dlib:linear}

\subsection{Maszyna wektorów nośnych}
\subsection{Sieci neuronowe}
\subsection{Brzegowa regresja jądra}

\section{Metody analizy modeli}

\subsection{Pole pod wykresem krzywej charakterystycznej odbiornika}

Biblioteka Dlib posiada implementację funkcji wyznaczającej krzywą ROC, jednak wymaga ona pewnego przetwarzania danych przed i po jej użyciu. W celu jej zastosowania należy podzielić dane sklasyfikowane prawidłowo i nieprawidłowo. Wynikiem funkcji jest wektor zawierający współrzędne poszczególnych punktów krzywej charakterystycznej odbiornika, które pozwalają na narysowanie wykresu na płaszczyźnie dwuwymiarowej. Obliczenie wartości pola pod wykresem należy dokonać ręcznie, wykorzystując np. jedną z numerycznych metod na obliczenie całki. Listing \ref{dlib:eval} przedstawia funkcję ewaluującą predykcje modelu, w tym obliczenie wartości pola pod wykresem krzywej charakterystycznej odbiornika dla zadania klasyfikacji.

\cppcode{Result/inc/dlib/eval.hpp}{Obliczenie pola pod wykresem funkcji ROC dla Dlib}{dlib:eval}

\subsection{Sprawdzian krzyżowy K-krotny}

\section{Dostępność dokumentacji i źródeł wiedzy}

Dlib posiada zbiór przykładów w postaci listingów kodów źródłowych realizujących poszczególne mechanizmy, dostępnych na stronie głównej projektu \cite{dlib:home}. Jest ona także jedną z głównych bibliotek omawianych w ramach wspomnianej wcześniej książki ,,Hands-On Machine Learning with C++''. Niestety większość forów społecznościowych skupia się na pracy z Dlib z poziomu interfejsu języka Python, co może utrudnić szukanie rozwiązań dla specyficznych przypadków. Warto wspomnieć, że oprócz funkcjonalności uczenia maszynowego, Dlib realizuje także inne zadania, jak np. networking, co sprawia, że przykłady kodów źródłowych dla programów machine learningu zgrupowane są razem z innymi mechanizmami. 