\chapter{Biblioteka Dlib}

\section{Wprowadzenie}

Jest to biblioteka do uczenia maszynowego napisana w nowoczesnym C++, o zastosowaniu przemysłowym oraz naukowym. Podobnie jak poprzednio omawiane biblioteki, posiada ona otwarte źródło na licencji Boost Software Licence \cite{dlib:license}. Do dziedzin wykorzystujących wyżej wspomnianą bibliotekę należą robotyka, systemy wbudowane, telefony komórkowe oraz śrowodiska o dużej wydajności obliczeniowej. Kod źródłowy biblioteki opatrzony jest testami jednostkowymi, co pozwala na łatwiejsze utrzymanie jakości dostarczanego rozwiązania. Ciekawym aspektem jest fakt, że Dlib stanowi nie tylko bibliotekę, lecz zestaw narzędzi, oferujący funkcjonalności wykraczające także poza dziedzinę uczenia maszynowego.

\section{Formaty źródeł danych}

Do reprezentacji wektora w bibliotece Dlib wykorzystywane są kontenery z biblioteki szablonów STL języka C++. Dodatkowo, istnieje możliwość ich inicjalizacji za pomocą operatora przecinka, oraz opakowania surowej tablicy (ang. \textit{raw array}). Oznacza to, że podobnie jak w przypadku biblioteki Shogun, dane mogą być przekazywane do programu wykorzystującego Dlib w dowolny sposób zapewniający umieszczenie ich np. w surowej tablicy do późniejszego przetworzenia na obiekty akceptowane przez bibliotekę. Metoda ta działa także z kontenerami biblioteki STL, które pozwalają na dostęp do surowych danych przy użyciu metody \textit{data()}. Tak samo jak poprzednio, występuje tu wsparcie dla formatu CSV obwarowanego tymi samymi ograniczeniami co dla Shogun. Za wspomniane wsparcie odpowiada przeładowany operator strumienia współpracujący z klasą \textit{std::ifstream} biblioteki standardowej C++. Przykładowy kod wykorzystujący opisany mechanizm zamieszczony został na listingu \ref{dlib:csv}.

\cppcode{Rozdzial6/csv.cpp}{Fragment kodu ilustrujący sposób odczytu z pliku w formacie CSV \cite{handsOnMachineLearning}}{dlib:csv}

\section{Metody przetwarzania i eksploracji danych}

\subsection{Normalizacja}
Biblioteka udostępnia normalizację danych poprzez standaryzację, realizowaną przez klasę \textit{Dlib::vector\texttt{\char`_}normalizer}. Głównym warunkiem ograniczającym zastosowanie jej jest fakt, że nie można w niej umieścić całego zestawu danych treningowych na raz, co wymusza podział obserwacji na osobne wektory, a następnie umieszczenie ich w kontenerze \textit{std::vector} do dalszego przetwarzania.

\subsection{Redukcja wymiarowości}

\subsubsection{PCA}
\subsubsection{Liniowa analiza dyskryminacyjna}
\subsubsection{Mapowanie Sammona}

\subsection{Sprawdzian krzyżowy K-fold}


\section{Modele uczenia maszynowego}

\subsection{Regresja liniowa}
\subsection{Brzegowa regresja jądra}
\subsection{Maszyna wektorów nośnych}
\subsection{Sieci neuronowe}

\section{Metody analizy modeli}

???

\section{Dostępność dokumentacji i źródeł wiedzy}

Dlib posiada zbiór przykładów w postaci listingów kodów źródłowych realizujących poszczególne mechanizmy, dostępnych na stronie głównej projektu \cite{dlib:home}. Jest ona także jedną z głównych bibliotek omawianych w ramach wspomnianej wcześniej książki ,,Hands-On Machine Learning with C++''. Niestety większość forów społecznościowych skupia się na pracy z Dlib z poziomu interfejsu języka Python, co może utrudnić szukanie rozwiązań dla specyficznych przypadków. Warto wspomnieć, że oprócz funkcjonalności uczenia maszynowego, Dlib realizuje także inne zadania, jak np. networking, co sprawia, że przykłady kodów źródłowych dla programów machine learningu zgrupowane są razem z innymi mechanizmami. 

\section{Przykłady testowe}

\subsection{Maszyna wektorów nośnych}
\subsection{Sieć neuronowa}